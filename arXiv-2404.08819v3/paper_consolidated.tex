%%%%%%%%%%%%%%%% BEGIN ICML FORMAT %%%%%%%%%%%%%%%%%%
% ICML Submission #: 6780

\documentclass{article}

% Recommended, but optional, packages for figures and better typesetting:
\usepackage{microtype}
\usepackage{graphicx}
\usepackage{subfigure}
\usepackage{booktabs} % for professional tables

% hyperref makes hyperlinks in the resulting PDF.
% If your build breaks (sometimes temporarily if a hyperlink spans a page)
% please comment out the following usepackage line and replace
% \usepackage{icml2024} with \usepackage[nohyperref]{icml2024} above.
\usepackage{hyperref}


% Attempt to make hyperref and algorithmic work together better:
\newcommand{\theHalgorithm}{\arabic{algorithm}}

% Use the following line for the initial blind version submitted for review:
\usepackage[accepted]{icml2024}

% If accepted, instead use the following line for the camera-ready submission:
% \usepackage[accepted]{icml2024}

% For theorems and such
\usepackage{amsmath}
\usepackage{amssymb}
\usepackage{mathtools}
\usepackage{amsthm}

% if you use cleveref..
\usepackage[capitalize,noabbrev]{cleveref}

%%%%%%%%%%%%%%%%%%%%%%%%%%%%%%%%
% THEOREMS
%%%%%%%%%%%%%%%%%%%%%%%%%%%%%%%%
\theoremstyle{plain}
\newtheorem{theorem}{Theorem}[section]
\newtheorem{proposition}[theorem]{Proposition}
\newtheorem{lemma}[theorem]{Lemma}
\newtheorem{corollary}[theorem]{Corollary}
\theoremstyle{definition}
\newtheorem{definition}[theorem]{Definition}
\newtheorem{assumption}[theorem]{Assumption}
\theoremstyle{remark}
\newtheorem*{remark}{Remark}
\newtheorem{example}[theorem]{Example}

% Todonotes is useful during development; simply uncomment the next line
%    and comment out the line below the next line to turn off comments
%\usepackage[disable,textsize=tiny]{todonotes}
\usepackage[textsize=tiny]{todonotes}

\icmltitlerunning{The Illusion of State in State-Space Models}
% The \icmltitle you define below is probably too long as a header.
% Therefore, a short form for the running title is supplied here:
% \icmltitlerunning{The Illusion of State in State-Space Models}


%%%%% ADDED MACROS, etc.
\input{preamble}

% \addtolength{\abovecaptionskip}{1ex}
% \addtolength{\belowcaptionskip}{-1ex}
% \addtolength{\floatsep}{-1ex}
% \addtolength{\textfloatsep}{-1.5ex}

\begin{document}

\twocolumn[
\icmltitle{The Illusion of State in State-Space Models}

% It is OKAY to include author information, even for blind
% submissions: the style file will automatically remove it for you
% unless you've provided the [accepted] option to the icml2024
% package.

% List of affiliations: The first argument should be a (short)
% identifier you will use later to specify author affiliations
% Academic affiliations should list Department, University, City, Region, Country
% Industry affiliations should list Company, City, Region, Country

% You can specify symbols, otherwise they are numbered in order.
% Ideally, you should not use this facility. Affiliations will be numbered
% in order of appearance and this is the preferred way.
% \icmlsetsymbol{equal}{*}

\begin{icmlauthorlist}
\icmlauthor{William Merrill}{nyu}
\icmlauthor{Jackson Petty}{nyu}
\icmlauthor{Ashish Sabharwal}{ai2}
\end{icmlauthorlist}

\icmlaffiliation{nyu}{New York University}
\icmlaffiliation{ai2}{Allen Institute for AI}

\icmlcorrespondingauthor{William Merrill}{willm@nyu.edu}
\icmlcorrespondingauthor{Jackson Petty}{petty@nyu.edu}
\icmlcorrespondingauthor{Ashish Sabharwal}{ashishs@allenai.org}

% You may provide any keywords that you
% find helpful for describing your paper; these are used to populate
% the "keywords" metadata in the PDF but will not be shown in the document
\icmlkeywords{Machine Learning, ICML}

\vskip 0.3in
]

% this must go after the closing bracket ] following \twocolumn[ ...

% This command actually creates the footnote in the first column
% listing the affiliations and the copyright notice.
% The command takes one argument, which is text to display at the start of the footnote.
% The \icmlEqualContribution command is standard text for equal contribution.
% Remove it (just {}) if you do not need this facility.

\printAffiliationsAndNotice{}  % leave blank if no need to mention equal contribution
% \printAffiliationsAndNotice{\icmlEqualContribution} % otherwise use the standard text.

%%%%%%%%%%%%%%%% END ICML FORMAT %%%%%%%%%%%%%%%%%%

\begin{abstract}
    % State-space models (SSMs) have emerged as a potential alternative architecture for building large language models (LLMs) compared to the previously ubiquitous transformer architecture. 
    State-space models (SSMs) have emerged as a potential alternative to transformers.
    One theoretical weakness of transformers is that they cannot express certain kinds of sequential computation and state tracking \citep{merrill-sabharwal-2023-parallelism}, which SSMs are explicitly designed to address via their close architectural similarity to recurrent neural networks. \emph{But do SSMs truly have an advantage (over transformers) in expressive power for state tracking?} Surprisingly, the answer is no. Our analysis reveals that the expressive power of S4, Mamba, and related SSMs is limited very similarly to transformers (within $\TC^0$), meaning
    % : they cannot express computation outside the complexity class $\TC^0$. In particular, this means
    these SSMs cannot solve simple state-tracking problems like permutation composition and consequently are provably unable to accurately track chess moves with certain notation, evaluate code, or track entities in a long narrative. To supplement our formal analysis, we report experiments showing that S4 and Mamba indeed struggle with state tracking. Thus, despite their recurrent formulation, the ``state'' in common SSMs is an illusion: S4, Mamba, and related models have similar expressiveness limitations to non-recurrent models like transformers, which may fundamentally limit their ability to solve real-world state-tracking problems. Moreover, we show that only a minimal change allows SSMs to express and learn state tracking, motivating the development of new, more expressive SSM architectures.
    % \footnote{\url{https://jpetty.org/ssm-illusion}}
\end{abstract}


\section{Introduction}


\begin{figure}[t]
    \centering
    \resizebox{.38\textwidth}{!}{
        \chessboard[
            setpieces={qa8, qb8, rc8, qd8, qe8, Pa2, Pb2, Ka1, kh1},
            addpgf={
                \tikz[overlay]\draw[move1red,line width=0.1em,->](c8) to[out=30,in=30] (c6);
                \tikz[overlay]\draw[move2red,line width=0.1em,->](c6)--(a6);
                \tikz[overlay]\draw[move3red,line width=0.1em,->](a6) to[out=150,in=210] (a8);
                \tikz[overlay]\draw[move1,line width=0.1em,->](a8)--(a7);
                \tikz[overlay]\draw[move2,line width=0.1em,->](a7)--(c7);
                \tikz[overlay]\draw[move3,line width=0.1em,->](c7)--(c8);
                \tikz[overlay]\draw[movegray,line width=0.1em,<->](a1)--(b1);
            },
        ]
    }
%     \begin{python}
% x = [0, 0, 1, 0, 0]
% x[1], x[3] = (x[3], x[1])  # Swap 1 and 3
%     \end{python}

    % Did it this way so box matches text.
    \noindent\fbox{\parbox{0.95\columnwidth}{%
        \small
        \pyth{x = [0, 0, 1, 0, 0]}\\
        \pyth{x[1], x[3] = x[3], x[1]  # Swap 1, 3}
    }}
    \vspace{0.5em}

    \noindent\fbox{\parbox{0.95\columnwidth}{%
        \emph{Alice, Bob, Carl, Dan, and Emma each have a coin. All are dimes except Carl's.
        Alice and Carl trade coins.}
    }}
    
    \caption{
    % \ashish{reduced size a bit; OK?}
    We prove that SSMs, like transformers, cannot solve inherently sequential problems like permutation composition ($S_5$), which lies at the heart of state-tracking problems like tracking chess moves in source-target notation  (see~\Cref{sec:chess}), evaluating Python code, or entity tracking. Thus, SSMs cannot, in general, solve these problems either.
        % \textit{Details:} In chess, transpositions become sequences of moves that swap pieces such as (\m{a8}, \m{a7}), (\m{a1}, \m{b1}), (\m{c8}, \m{c6}), (\m{b1}, \m{a1}), (\m{a7}, \m{c7}), (\m{a1}, \m{b1}), (\m{c6}, \m{a6}), (\m{b1}, \m{a1}), (\m{c7}, \m{c8}), (\m{a1}, \m{b1}), (\m{a6}, \m{a8}), (\m{b1}, \m{a1}).
    % In Python, transpositions become tuple assignments that swap values.
    \begin{tabular}{c} \\
         \hspace{0.7em}\textbf{Code:} \url{http://jpetty.org/ssm-illusion}
    \end{tabular}
    % \begin{center}
    %     \textbf{Code:} \url{http://jpetty.org/ssm-illusion}
    % \end{center}
    }
    \label{fig:s5-chess-code}
\end{figure}


Recent theoretical work has shown that transformer architecture based models are incapable of expressing inherently sequential computation \citep{merrill-sabharwal-2023-parallelism}. These results reveal a surprising limitation of transformers: they cannot express simple kinds of \emph{state tracking} problems, such as composing sequences of permutations, which even simple recurrent neural networks (RNNs) can naturally express.
In a different line of work, state space model (SSM) architectures \citep{gu2021combining,gu2022efficiently,fu2023hungry,gu2023mamba,wang2024mambabyte} have been introduced as an alternative to transformers, with the goal of achieving RNN-like expressive power for handling problems that are naturally stateful and sequential \citep{gu2021combining,gu2022blog}.
\emph{But does the seemingly stateful design of SSMs truly enable them to solve sequential and state-tracking problems that transformers cannot?}
If so, this would be a promising property of SSMs because state tracking is at the heart of large language model (LLM) capabilities such as tracking entities in a narrative \citep{heim1983file,kim-schuster-2023-entity}, playing chess under certain notation\footnote{The hardness of chess state tracking holds with (source, target) notation, but standard notation may make state tracking easier.}, or evaluating code. This would motivate further research on SSM architectures and their deployment in the next generation of LLMs.

% \will{Could include block quote from \citet{gu2021combining} and position this paper as responding to it:
% ``Ideally, a model family would combine the strengths of these paradigms, providing properties like parallelizable training (convolutional), stateful inference (recurrence) and time-scale adaptation (differential equations), while handling very long sequences in a computationally efficient way.''
% \url{https://arxiv.org/pdf/2110.13985.pdf}
% }
% Chris Re's group's blog post: https://hazyresearch.stanford.edu/blog/2022-01-14-s4-3
% Two advantages of RNNs: statefulness, fast inference
% Mentions potential that SSMs are less expressive but hedges possibility
% Sasha's Blog post: https://srush.github.io/annotated-s4/#part-1-state-space-models

In this work, we show that the apparent stateful design of SSMs is an \emph{illusion} as far as their expressive power is concerned. In contrast to the suggestion by \citet{gu2021combining,gu2022blog} (and, perhaps, a broader belief in the community) that SSMs have expressive power for state tracking similar to RNNs, we prove theoretically that linear and Mamba-style SSMs, like transformers, cannot express inherently sequential problems, including state-tracking problems like composing permutations that RNNs can easily express. Further, our experiments confirm this prediction: both transformers and these SSMs cannot learn to compose permutations with a fixed number of layers, whereas RNNs can compose permutations with just a single layer.
Our results imply that arguments that current SSMs have an advantage over transformers due to being ``more recurrent'' or capable of tracking state are misguided. In fact, the SSM architectures we consider are just as theoretically unequipped for state tracking and recurrent computation as transformers are.

We first establish the theoretical weakness of linear SSMs and near generalizations by proving they are in the complexity class $\L$-uniform $\TC^0$, which has been previously shown for transformers \citep{merrill-sabharwal-2023-parallelism}. This implies these SSMs cannot solve inherently sequential problems (formally, problems that are $\NC^1$-hard), including state-tracking problems like permutation composition \citep{liu2023transformers}. Permutation composition is a fundamental problem at the heart of many real-world state-tracking problems such as playing chess, evaluating code, or tracking entities in a narrative (\Cref{fig:s5-chess-code}), implying solutions to these problems, too, cannot be expressed by SSMs, at least in the worst case.

At first glance, our results may appear to contradict \citet{gu2021combining}'s claim that linear SSMs can simulate general recurrent models, which can express permutation composition. But the contradiction is resolved by a difference in assumptions: \citet{gu2021combining} relied on \emph{infinite depth}
% \ashish{do we really mean infinite? or finite but unbounded (as a function of input length)?} \will{I believe they really mean infinite}
(number of layers) to show that SSMs could simulate RNNs. We, on the other hand, analyze the realistic setting with a bounded number of layers, under which we find that SSMs cannot simulate the recurrent state of an RNN and, in fact, suffer from similar limitations as transformers for state tracking.

Empirically, we find that S4 \citep{gu2022efficiently} and S6 \citep{gu2023mamba} SSMs, as well as transformers, do \emph{not} learn to solve the permutation composition state-tracking problem with a fixed number of layers, while simple RNNs can do so with just one layer. This provides empirical support for our theoretical separation in expressive power for state tracking between SSMs and true recurrent models. We also find that both transformers and SSMs struggle compared to RNNs on state-tracking problems less complex than permutation composition where it is not known whether they can express a solution.
% , but, intuitively, which may be difficult to learn or express due to its non-commutative nature.
Thus, in practice, SSMs may struggle not just on the hardest state-tracking problems like permutation composition but also on easier variants.

\begin{figure}
    \centering
    \begin{tikzpicture}
    % Complexity classes.
    \node[
          above,ellipse,draw,
          minimum height=8em,
          minimum width=18em,
          fill=lightred,
         ] (nc1) {};
    \node[
          above,ellipse,draw,
          minimum height=6em,
          minimum width=14em,
          fill=lightpurple,
         ] (tc0) {};

    % NC1-complete problems.
    \node[below left=0.25cm and 0.7cm of nc1.north,font=\footnotesize] {\textit{Chess}};
    \node[below left=0.6cm and 2.0cm of nc1.north,font=\footnotesize] {$S_5$};
    \node[below right=0.25cm and 0.7cm of nc1.north,font=\footnotesize] {\textit{Code}};
    \node[below right=0.6cm and 1.6cm of nc1.north,font=\footnotesize] {\textit{Entities}};

    % Models in TC0.
    \node[
          left = 1.2cm of tc0.center,
          font=\footnotesize,
          anchor=center,
         ] {SSMs};
    \node[
          right = 1cm of tc0.center,
          font=\footnotesize,
          anchor=center,
         ] {Transformers};
    % \node[
    %       above = 0.025cm of tc0.center,
    %       font=\footnotesize,
    %       text width=3.5cm
    %      ] {\underline{SSMs (Thms.~\ref{thm:non-gated}~\&~\ref{thm:diagonal})}};
    % \node[
    %       above = -0.8cm of tc0.center,
    %       font=\footnotesize,
    %       text width=3.5cm,
    %      ] {Transformers~\citep{merrill-sabharwal-2023-parallelism}};

    % Class labels.
    \path
        (tc0.north) node[below] {$\TC^0$}
        (nc1.north) node[below] {$\NC^1$};
    \end{tikzpicture}
    \caption{Complexity hierarchy within $\NC^1$. Transformers can only recognize languages within $\TC^0$ \citep{merrill-sabharwal-2023-parallelism}, and we show the same for SSMs (\Cref{thm:non-gated,thm:diagonal}). Thus, both architectures cannot express the ``hard state tracking'' captured by $\NC^1$-complete problems like $S_5$, which \emph{can} be straightforwardly expressed by RNNs. The figure assumes the widely held conjecture $\TC^0 \neq \NC^1$.}
    \label{fig:hierarchy}
\end{figure}

Finally, we consider a minimal extension of a linear SSM which makes the transition matrix input dependent, similar to Liquid S4 \cite{hasani2023liquid}. We show that this extension has sufficient expressive power for state tracking and permutation composition. Empirically, we show that our implementation of this extension learns to solve permutation composition with a single layer, just like an RNN, while being similarly parallelizable to other SSMs.
It is an open question whether such SSM architectures with greater expressivity for state tracking are practically viable for large-scale language modeling.

\section{Background}

We first present the SSM architectures we will analyze (\Cref{sec:ssms}).
Our analysis of the state tracking capabilities of SSMs borrows deeply from the circuit complexity and algebraic formal language theory literature. We thus review how circuit complexity can be used to analyze the power of neural networks (\Cref{sec:circuits}) and how state-tracking problems can be captured algebraically and analyzed within the circuit complexity framework (\Cref{sec:state-tracking}).

\subsection{Architecture of State-Space Models} \label{sec:ssms}

SSMs are a neural network architecture for processing sequences similar in design to RNNs or linear dynamical systems.
SSMs have been suggested to have two potential advantages compared to transformers owing to their recurrent formulation: faster inference and, possibly, the ability to better express inherently sequential or stateful problems \citep{gu2021combining,gu2022blog}.
Several architectural variants of SSMs have been proposed, including S4 \citep{gu2022efficiently} and Mamba \citep{gu2023mamba}.
Recently, SSMs have been shown to achieve strong empirical performance compared to transformers in certain settings, particularly those involving a long context \citep{gu2023mamba,wang2024mambabyte}.

SSMs consist of \emph{SSM layers}, which can be thought of as simplified RNN layers. We define a \emph{generalized linear SSM layer} that encapsulates both S4 \citep{gu2022efficiently} and the S6 layer used by Mamba \citep{gu2023mamba} as special cases.

\begin{definition}[Generalized linear SSM layer]
    \label{def:generalized-linear-ssm}
    Given a sequence\footnote{In practice, an SSM is often applied elementwise ($k=1$) on each feature $1, \ldots, m$ of hidden state $\mathbf x_1, \ldots, \mathbf x_n \in \mathbb R^m$.} $\mathbf x_1, \ldots, \mathbf x_n \in \mathbb R^k$, the \emph{recurrent form} of a linear SSM layer defines a new sequence of states $\mathbf h_1, \ldots, \mathbf h_n \in \mathbb R^d$ using projections $\barA_i \in \mathbb R^{d \times d}$ and $\barB_i \in \mathbb R^{d \times k}$, which can themselves depend on $\mathbf x_i$. For each $1 \leq i \leq n$,
    \begin{equation}
        \label{eqn:recurrent}
        \mathbf h_i = \barA_i \mathbf h_{i-1} + \barB_i \mathbf x_i .
        \tag{Recurrent form}
    \end{equation}
    The \emph{convolutional form} of the SSM layer defines the same\footnote{The two forms express the same function over $\mathbb R$ or any other distributive datatype. Over floating points (\Cref{sec:datatype}), they are not guaranteed to be the same, but we must assume the error is negligible for them to be well-defined and usable in practice.} $\mathbf h_1, \ldots, \mathbf h_n$ computed differently as a summation:\footnote{We define the cumulative product to unroll from greatest to least, e.g., $\prod_1^3 \mathbf A_i = \mathbf A_3 \mathbf A_2 \mathbf A_1$. The order is important due to the non-commutativity of matrix multiplication.}
    \begin{equation}
        \label{eqn:convolutional}
        \mathbf h_i = \sum_{j=1}^i \left( \prod_{k=j+1}^i \barA_k \right) \barB_j \mathbf x_j .
        \tag{Convolutional form}
    \end{equation}
    The layer outputs $\mathbf y_i = \mathbf C_i \mathbf h_i + \mathbf D_i \mathbf x_i \in \mathbb R^k$, where $\mathbf C_i \in \mathbb R^{k \times d}$ and $\mathbf D_i \in \mathbb R^{k \times k}$ depend on $\mathbf x_i$.
\end{definition}

Two common cases of this layer are when $\barA_i$ does not depend on the input (``non-gated''; \Cref{sec:non-gated}) and when $\barA_i$ is diagonal (\Cref{sec:diagonal}). In both of these cases, we will show that the SSM can be simulated in $\TC^0$.

A \textbf{generalized linear SSM} is made up of multiple such layers, with a linear projection and a non-linearity applied after every layer \citep{rush2022s4}. Layer-norm can also be applied, either before or after the layer.

\paragraph{Practical Details.}
In S4 and related SSMs, \Cref{def:generalized-linear-ssm} is applied elementwise ($k=1$) across all $m$ elements of the previous layer output \citep{gu2022efficiently}. In practice, the weight matrix initialization is crucial for training.
Our expressivity results (\Cref{thm:non-gated,thm:diagonal}) apply for any generalized linear SSM (including S4 and S6), independent of initialization.
In contrast to S4 and S6, H3 \citep{fu2023hungry} does not meet \Cref{def:generalized-linear-ssm} because the context is not represented by a single vector. Rather, it resembles a transformer with SSM components.


\subsection{Numeric Datatype}
\label{sec:datatype}

Circuit-complexity analysis of neural networks depends to some degree on low-level details about arithmetic and the underlying datatype $\D$ used in the network's computation graph. We can think of $\D$ as parameterized by the number of bits available to represent a number in $\D$. For instance, non-negative integers in $[0, 2^p]$ use $p$ bits, signed integers in $[-2^p, 2^p]$ use $p+1$ bits, FP16 uses 16 bits, etc.

Our main results (\Cref{thm:non-gated,thm:diagonal}) will go through for any datatype $\D$ for which the following operations are efficiently parallel-computable (i.e., are in the complexity class $\L$-uniform $\TC^0$, to be defined shortly in \cref{sec:circuits}):
%
\begin{compactenum}
    \item Iterated addition, i.e., summing $n$ numbers in $\D$
    \item Iterated product, i.e., multiplying $n$ numbers in $\D$
    \item Matrix powering, i.e., computing the $n$-th power of a fixed-size $d \times d$ matrix over $\D$
\end{compactenum}

When $\D$ is any finite-precision datatype, i.e., has a fixed number of bits available (e.g., 16 or 64), then these operations are easily seen to be in $\L$-uniform $\TC^0$. As \citet{merrill2023logic} argue, however, finite-precision datatypes severely limit the expressivity of neural architectures from a formal perspective (e.g., finite-precision transformers cannot represent uniform attention), motivating the use of parameterized datatypes that can (approximately) represent any number with a sufficiently large parameter. Interestingly, when $\D$ is the datatype of $n$-bit integers, all of the above operations are known to be in $\L$-uniform $\TC^0$ \cite{hesse2001division,mereghetti2000threshold}. Realistically, however, neural model implementations use floating point numbers with much fewer than $n$ bits. Following \citet{merrill2023logic}, we use the \textbf{log-precision floating point} model, i.e., $c \log n$ bit floats where $c$ is some fixed constant (see \Cref{app:arithmetic} for a formal definition). \citet{merrill-sabharwal-2023-parallelism} showed that iterated addition over log-precision floats is in $\L$-uniform $\TC^0$. We extend the arguments of \citet{hesse2001division} and \citet{mereghetti2000threshold} to show that iterated product and matrix powering over log-precision floats are also in $\L$-uniform $\TC^0$ (see \Cref{app:arithmetic}).


\subsection{Limits of Transformers via Circuit Complexity} \label{sec:circuits}

A line of recent work has used circuit complexity and logic formalisms to identify expressivity limitations of transformers on reasoning problems \cite{angluin-2023-BRASP,merrill-sabharwal-2023-parallelism,liu2023transformers,chiang2023tighter,merrill2023logic,hao2022ac0}; see \citealp{strobl2023transformers} for a survey. In particular, \citet{merrill-sabharwal-2023-parallelism} showed transformers can only solve problems in the complexity class $\TC^0$, which is the set of problems that can be recognized by constant-depth, polynomial-size threshold circuit families. Such circuits, in addition to having standard AND, OR, and NOT gates (of arbitrary fan-in), can also use threshold gates that output $1$ iff at least $k$ of the inputs are $1$, where $k$ is a parameter of the gate. Informally, $\TC^0$ can be thought of as the class of problems that can be solved with extremely parallel (constant-depth) computation.\footnote{We use $\TC^0$ to mean $\L$-uniform $\TC^0$, meaning the circuit family is constructible by a Turing machine that runs in space logarithmic in the size of the input \citep[cf.][]{merrill-sabharwal-2023-parallelism,strobl2023transformers}.
We believe our results could be extended from $\L$-uniform $\TC^0$ to $\mathsf{DLOGTIME}$-uniform $\TC^0$ using techniques similar to \citet{merrill2023logic} for composing $\TC^0$ circuits in a way that preserves $\mathsf{DLOGTIME}$ uniformity.
}

Problems outside $\TC^0$, corresponding to problems that are inherently sequential and thus cannot be parallelized, cannot be solved by transformers. No problems in polynomial time are known unconditionally to be outside $\TC^0$, but unless the widely held conjecture that $\TC^0 \neq \NC^1$ is false, many simple $\NC^1$-hard problems are outside $\TC^0$. In particular, this includes simulating finite automata ($\NC^1$-complete), evaluating boolean formulas ($\NC^1$-complete), determining graph connectivity ($\L$-complete), and solving linear equations ($\P$-complete). These problems have already been shown to be inexpressible by transformers \citep{merrill-sabharwal-2023-parallelism}. By showing that SSMs can be simulated in $\TC^0$, we will establish that they also cannot be solved by SSMs.

\section{State Tracking} \label{sec:state-tracking}

Informally, a state-tracking problem is a problem where the text specifies some sequence of updates to the state of the world, and the goal of the problem is to determine what the world state is after the updates have been applied in sequence.
The circuit complexity view on the power of neural networks can be combined with other insights from algebraic formal language theory to analyze the kinds of state tracking that SSMs can express.
In particular, this theory reveals which kinds of state-tracking problems are (likely) not in $\TC^0$. This will, in turn, allow us to find examples of hard state tracking that models like SSMs cannot express.

\subsection{State Tracking as a Monoid Word Problem} \label{sec:word-problems}

 From the perspective of algebraic formal language theory, state tracking over a finite world can be captured as a \emph{word problem} on a \emph{finite monoid} \citep{liu2023transformers}.\footnote{We consider finite monoids for simplicity, but the approach may be extendable to infinite (e.g., finitely generated) monoids.}
Different updates to the world become different elements in the monoid, and resolving the final world state after all the updates have been applied is equivalent to computing the product of a sequence of elements (also called a ``word'').

\begin{definition}[Word problem]
    Let $M$ be a finite set, and $(M,\cdot)$ a finite monoid (i.e., $M$ with identity and associative multiplication). The word problem for $M$ is 
    % to express the induced homomorphism from the free monoid $M^*$ into $M$ which
    to reduce sequences in $M^*$ under multiplication; that is, send $m_0 m_1 \cdots m_k$ to $m_0 \cdot m_1 \cdot \ldots \cdot m_k \in M$. Solving the word problem requires reducing sequences of arbitrary length.
\end{definition}

\begin{example} \label{ex:parity}
Consider the monoid $\{0, 1\}$ where $\cdot$ is addition modulo 2. The word problem is to compute the parity of a string, e.g., $0011 \mapsto 0$. From a state-tracking perspective, this monoid captures a world with a single light switch. Identity $0$ corresponds to no action, and $1$ flips the switch.
\end{example}

% Formally, let $M$ be a finite set, and $(M,\cdot)$ a monoid. The word problem for $M$ is to learn the induced homomorphism from $M^* \to M$ which reduces sequences of elements under multiplication, i.e., which sends $m_0 m_1 \cdots m_k$ to $m_0 \cdot m_1 \cdots m_k \in M$. Solving the word problem requires reducing sequences of arbitrary length.

% and $M^* \coloneqq (M,\circ)$ be the free monoid on $M$, which contains sequences of elements from $M$ composed under a concatenation operation $\circ$. Let $(M,\cdot)$ be a group likewise defined on $M$. The word problem for any such monoid is to learn the mapping $\phi\colon M^* \to (M,\cdot)$ which sends each sequence in $s = m_0 \circ m_1 \circ \cdots \circ m_k$ to its reduction via group multiplication $m_0 \cdot m_1 \cdots m_k = m \in M$.

Modeling state tracking with word problems lets us draw connections between circuit complexity and algebra to understand which word problems are hard to solve. \citet{krohn1965algebraic} established that not all word problems are created equal: some, like \Cref{ex:parity}, are in $\TC^0$, while others are $\NC^1$-complete, requiring recurrent processing to solve \citep{immerman1989complexity,barrington1989bounded}. Because we will show SSMs can be simulated in $\TC^0$, it follows that $\NC^1$-complete state-tracking problems cannot be expressed by SSMs (cf.~\Cref{fig:hierarchy}).

Whether or not a word problem is $\NC^1$-complete depends on the algebraic structure of the underlying monoid. \citet{barrington1989bounded} showed that the word problem of every finite non-solvable\footnote{We focus on word problems on groups, which are monoids with inverses. Formally, a group $G$ is solvable exactly when there is a series of subgroups $1 = G_0 < G_1 < \cdots < G_k = G$ such that $G_{i-1}$ is normal in $G_i$ and $G_i/G_{i-1}$ is abelian.} group is  $\NC^1$-complete. That non-solvable groups have $\NC^1$-complete word problems is notable because of the ubiquity with which non-solvable groups show up in tasks involving state tracking. The canonical example of an $\NC^1$-complete word problem is that of $S_5$, the symmetric group on five elements that encodes the permutations over five objects. As an immediate instantiation of this, consider a document describing a sequence of transpositions: \emph{``swap ball 1 and 3, swap ball 3 and 5, swap ball 4 and 2, ...''}.
Being able to answer the question \emph{``where does ball 5 end up?''} for all possible swap sequences requires solving the $S_5$ word problem.\footnote{W.l.o.g., any permutation can be factored into a sequence of transpositions, or swaps.} 
% Thus transpositions over five elements are a generator for $S_5$.}
Beyond permutations, \Cref{fig:s5-chess-code} shows how many natural state-tracking problems like tracking chess moves, evaluating code, or tracking entities also encode the structure of $S_5$, meaning these state-tracking problems also cannot be expressed by a model in $\TC^0$. Rather, in order to solve these problems, the depth of the model would have to be expanded to accommodate longer inputs.

Although the $S_5$ word problem is canonical, in this paper we will consider the word problem on a closely related group $A_5$: the \emph{alternating} group on five elements. We do this for simplicity: $A_5$ is a subgroup of $S_5$ containing only even permutations, and is the smallest non-solvable subgroup. We will compare the word problem on $A_5$ to two other baseline groups: $A_4 \times \mathbb{Z}_5$, a non-abelian but solvable group; and $\mathbb{Z}_{60}$, an abelian group encoding mod-$60$ addition. We choose these groups as points of comparison because they all have $60$ distinct elements, meaning that the difficulty in learning their word problems will come only from the complexity of learning the group multiplication operation.

\subsection{Encoding $S_5$ in Chess State Tracking} \label{sec:chess}

% \jackson{Don't we care about showing the other direction? I.e., that we can embed $S_5$ in chess?} \will{Think this is just a terminology thing... in complexity, reducing $S_5$ to chess means that, given an $S_5$ instance, we can come up with a chess game encoding it. ie if we have a subroutine for chess, we can solve $S_5$.} \jackson{ahh okay!}

\Cref{fig:s5-chess-code} already gives some intuition into how state-tracking problems encode $S_5$. Out of these examples, the most intricated case is chess. We now give a proper reduction from $S_5$ to tracking chess moves, showing formally that not just $S_5$, but chess state tracking as well, is $\NC^1$-complete.
We define the chess state-tracking problem as follows:

\begin{compactitem}
    \item \textbf{Input:} A \textbf{chessboard state} and \textbf{sequence of chess moves}, where each move is written in UCI notation as a tuple (source square, target square). This differs from the standard SAN notation that represents other information like piece type \citep{Toshniwal2021ChessAA}.
    \item \textbf{Output:} The resulting board state after starting in the initial board state and applying the sequence of moves one after another, ignoring draws. If any move is illegal given the previous board state, a null state is returned.
\end{compactitem} 

We show that $S_5$ can be reduced to chess state tracking, establishing its $\NC^1$-completeness:
% That is, we can map any $S_5$ sequence to a sequence of chess moves and read off the answer to the $S_5$ instance from the final chessboard state.

\begin{proposition}\label{prop:chess}
    $S_5$ can be reduced to chess state tracking in UCI notation via $\NC^0$ reductions.
\end{proposition}

\begin{proof}
    Without loss of generality, we consider the variant of $S_5$ where the output is true if and only if the original first element returns to the first position after the given sequence of permutations has been applied.
    
    The idea, as illustrated in \Cref{fig:s5-chess-code}, is to map each element of $S_5$ to a fixed sequence of chess moves that permutes five pieces accordingly on the chessboard.
    % Then, the final chessboard state will allow us to determine the composition of the permutation sequence.
    Given an instance of the $S_5$ word problem, we will construct an initial board state and a sequence of moves such that the final chessboard state encodes the output of that $S_5$ problem instance.

    Let $M$ denote the set of chess moves in the UCI, i.e., (source square, target square), notation.
    
    \noindent \textbf{Initial Board State.} We construct a chessboard similar to \Cref{fig:s5-chess-code} but with a black rook at $\m{a8}$ and black queens at $\m{b8}$ to $\m{e8}$.

    \noindent \textbf{Chess Move Sequence.}
    We then construct a finite function $f : S_5 \to M^*$ that encodes a permutation $\pi$ as a sequence of chess moves. We first factor each permutation $\pi$ to a sequence of transpositions $\tau_1(\pi) \cdots \tau_{m_\pi}(\pi)$.
    Each transposition $\tau \in T$ can in turn be expressed as a sequence of chess moves analogously to \Cref{fig:s5-chess-code}. For example, transposing items 1 and 3 can be expressed as the move sequence: (\m{a8}, \m{a7}), (\m{a1}, \m{b1}), (\m{c8}, \m{c6}), (\m{b1}, \m{a1}), (\m{a7}, \m{c7}), (\m{a1}, \m{b1}), (\m{c6}, \m{a6}), (\m{b1}, \m{a1}), (\m{c7}, \m{c8}), (\m{a1}, \m{b1}), (\m{a6}, \m{a8}), (\m{b1}, \m{a1}), which has the crucial property that it transposes \m{a8} with \m{c8}. We denote the mapping from transpositions to chess move sequences as $f : T \to M^*$. Putting it all together, we have
    \begin{equation*}
        f(\pi) = \bigcomp_{j=1}^{m_\pi} f(\tau_j(\pi)) .
    \end{equation*}
    To reduce a sequence of permutations $w \in S_5^*$, we let
    \begin{equation*}
        f(w) = \bigcomp_{i=1}^{n} f(w_i) .
    \end{equation*}

    \noindent \textbf{Putting It All Together.} We call our oracle for chess state tracking with the constructed initial board state and $f(w)$ as the sequence of chess moves. By construction, we can then return true if and only if the rook is at $\m{a8}$. The reduction can be implemented in $\NC^0$ because it is a simple elementwise mapping of the input tokens, and decoding from the output chessboard is a finite table lookup.
\end{proof}

As a fun aside, we note that the chess board constructed in the above proof is reachable in a standard chess game. The chess sequences encoding permutation sequences are all valid in the game of chess, except that they ignore the fact that repeated board states in chess technically lead to a draw.

Since $S_5$ is $\NC^1$-complete under $\AC^0$ reductions and $\NC^0 \subseteq \AC^0$, we have:

\begin{corollary}
    The chess state-tracking problem is $\NC^1$-complete under $\AC^0$ reductions.
\end{corollary}

Theorem 3.2 of \citet{feng2023towards} uses a similar reduction to prove formula evaluation is $\NC^1$-complete.
Reductions can be constructed for evaluating Python or tracking entities in a dialog, as suggested by \Cref{fig:s5-chess-code}.
As for chess, the task formatting for entity tracking affects its hardness. For instance, the formatting used by \citet{kim-schuster-2023-entity} in their Figure 1 is not $\NC^1$-complete, whereas the variant shown in our Figure 1 is. This underscores the value of theory for constructing examples of hard state tracking.

% This view allows us to import known theory from algebra to analyze which state-tracking problems are hard and which can be solved simply. Quite counterintuitively, classical theory \citep{krohn1965algebraic} establishes that not all monoid word problems have the same complexity. From a circuit complexity point of view, some require recurrent processing to solve since they are $\NC^1$-complete, while others are in $\TC^0$ \citep{immerman1989complexity,barrington1989bounded}. 




% The canonical example of an $\NC^1$-complete state-tracking problem captured as a monoid word problem is $S_5$: the word problem for the permutation group over five elements. \will{Should we mention that $S_4$ is easy?} Informally, an $S_5$ input is a sequence of permutations over five elements (e.g., \emph{``swap ball 1 and 3, swap ball 3 and 5, swap ball 4 and 2''}\footnote{Without less of generality, any permutation can be factored into a sequence of transpositions, or swaps. This means the transpositions over five elements are a generator for $S_5$.}). The goal of the problem is to determine which permutation of the elements is represented when all the input elements are composed, or, without loss of generality, to say where a particular ball ends up (e.g., \emph{``ball 1 ends up at position 5''}). One reason this problem is canonical is that its alternating subgroup is the smallest monoid whose word problem is $\NC^1$-complete \citep{barrington1989bounded}. It is also quite naturally embedded within real-world state-tracking problems like evaluating code or playing chess (\Cref{fig:s5-chess-code}) \ashish{it's not obvious that $S_5$ is embedded in chess} \will{will ref appendix}. $S_5$ thus serves as a good case study for evaluating the state tracking capabilities of SSMs and transformers.


%%%

% To examine how well state-space models learn tasks which require state tracking, we construct word problems on monoids associated with three finite groups. Let $(G, \cdot)$ be a group on a set $G$. Let $s \in G^+ = s_0 \cdot s_1 \cdots s_k$ be a sequence in the free monoid defined on $G$. The word problem on $G$ is then learning the mapping $G^+ \to G$ which maps $s$ to an element of $G$ be reducing $s$ via group multiplication. This can be seen as a generalization of learning some commonly-studied math tasks for neural networks (see, \emph{inter alia},  \citealt{graves2014neural, power2022grokking, hu2024latent}): learning arithmetic modulo $k$ is the same as learning the word problem for the cyclic group $\mathbb{Z}_k$. 

% Not all word problems are created equal; \citet{barrington1989bounded} showed that word problems for solvable groups can be solved with circuits in $\TC^0$, while those for non-solvable groups cannot. This means that for solvable groups, sequences of length $k$ can be reduced with circuits of constant depth, while for non-solvable groups sequences of length $k$ can only be reduced by circuits of depth logarithmic in $k$. In practice, this represents an upper bound on the expressivity of state-space models trained on a word problem task; actual performance may do worse than this if models have difficulty learning the task.

% We study whether or not this is the case by comparing the performance of three different kinds of neural models (RNNs, Transformers, and Mamba) on word problems on three different groups ($\mathbb{Z}_{60}, A_4\times \mathbb{Z}_5, A_5$) on sequences of lengths $3$ through $12$. We choose these groups because they each have 60 distinct elements, meaning that any differences in performance result from the models' abilities to learn the multiplication reduction and not the number of possible elements a sequence can reduce to. $\mathbb{Z}_{60}$ is an ``easy'' group to learn since it is commutative, so the order of multiplicands is not important. $A_4\times \mathbb{Z}_5$ is non-commutative but solvable, so it can be represented by $\TC^0$ circuits for sequences of arbitrary length. $A_5$ is the smallest non-solvable group\footnote{Formally, a group $G$ is solvable if there exists a sequence $1 = G_0 < G_1 < \cdots < G_k = G$ such that $G_{i-1}$ is normal in $G_i$ and $G_i/G_{i-1}$ is abelian. $A_5$ is a subgroup of $S_5$, and consists of all even permutations over $5$ objects.}, meaning that $\TC^0$ circuits of fixed depth can reduce sequences exponentially long in the depth of the circuit.

\section{SSMs Can be Simulated in $\TC^0$}

In this section, we show that the convolutional form of common variants of SSM can be simulated in $\TC^0$. Assuming the convolutional form of the model computes the same function as the recurrent form, this implies such SSMs cannot solve inherently sequential problems, despite their appearance of recurrence and statefulness. We first show containment in $\TC^0$ for non-gated SSMs (\Cref{thm:non-gated}), and then show the same holds for diagonal SSMs (\Cref{thm:diagonal}).
% The main idea in both proofs is that matrix powering can be computed in $\TC^0$ \citep{mereghetti2000threshold}, and computing the convolutional form of an SSM can essentially be reduced to matrix powering.

\subsection{Conditions for Linear SSMs in $\TC^0$}

Before characterizing specific SSM architectures, we first show that the complexity of computing transition matrix products essentially determines the complexity of simulating an SSM with a circuit family.

\begin{lemma} \label{lem:general}
    Let $M$ be a log-precision generalized linear SSM. Then there exists an $\L$-uniform $\TC^0$ circuit family that computes $M$'s convolutional form if:
    \begin{compactenum}
        \item For any integer interval $[j, k]$, the matrix product $\prod_{i = j}^k \barA_i$ can be computed in $\L$-uniform $\TC^0$ as a function of $\barA_j, \ldots, \barA_k$ (to $c \log n$ precision for any $c > 0$).
        \item For $1 \leq i \leq n$, $\barA_i, \barB_i, \mathbf C_i,$ and $\mathbf D_i$ can be computed in $\L$-uniform $\TC^0$ as a function of $\mathbf x_i$.
    \end{compactenum}
\end{lemma}

\begin{proof}
    Following the proof structure of \citet{merrill-sabharwal-2023-parallelism}, we describe how to construct a log-space bounded Turing machine $T_M$ that, given $\mathbf x_1, \ldots, \mathbf x_n$ as input, prints a circuit that simulates $M$ on this input. We first note that for all processing done before or after an SSM layer (projection, non-linearity, layer norm, etc.), $T_M$ can follow known simulations of such operations for transformers~\citep{merrill-sabharwal-2023-parallelism,merrill-sabharwal-2024-cot} to output a $\TC^0$ circuit simulating this processing. We thus focus on simulating an individual SSM layer.
    
    Recall from \cref{def:generalized-linear-ssm} that $M$'s convolutional form requires computing $\mathbf h_i = \sum_{j=1}^i \left( \prod_{k=j+1}^i \barA_k \right) \barB_j \mathbf x_j$ and $\mathbf y_i = \mathbf C_i \mathbf h_i + \mathbf D_i \mathbf x_i$. By the second precondition, $T_M$ can print a $\TC^0$ circuit that computes all matrices involved here. Further, by the first precondition, $T_M$ can also print a $\TC^0$ circuit that computes the innermost product in the computation of each hidden state $\mathbf h_i$, namely $\prod_{k=j+1}^i \barA_k$. It can now print a $\TC^0$ circuit to multiply the resulting product\footnote{Let $c \log n$ be the SSM's precision. We compute $\prod_k \barA_k$ to $c' \log n$ precision for a large enough $c'$ (similar to the proof of \cref{cor:iterated-float-product}) such that the full product $\left( \prod_k \barA_k \right) \barB_j \mathbf x_j$ is correct to at least $c \log n$ bits, as technically required by \Cref{def:flattening}.} with $\barB_j$ and $\mathbf x_j$, and then print a circuit to compute an iterated sum over the $i$ resulting vectors to compute $\mathbf h_i$ (cf.~iterated addition in \cref{app:arithmetic}). It can similarly print a (simpler) circuit to compute $\mathbf y_i$. Thus, the entire SSM layer can be simulated by an $\L$-uniform $\TC^0$ circuit.
\end{proof}

We will use \Cref{lem:general} to show that any non-gated or diagonal generalized linear SSM can be simulated in $\TC^0$.

\subsection{Non-Gated SSMs are in $\TC^0$} \label{sec:non-gated}

\begin{theorem}[Non-gated SSM] \label{thm:non-gated}
    Let $M$ be a log-precision generalized linear SSM such that, for any $i$,
    \begin{equation*}
        \barA_i = \barA, \quad
        \barB_i = \barB, \quad
        \mathbf C_i = \mathbf C, \quad
        \mathbf D_i = \mathbf D .
    \end{equation*}
    Then there exists an $\L$-uniform $\TC^0$ circuit family that computes $M$'s convolutional form.
\end{theorem}

\begin{proof}
    We prove this by showing that both conditions from \Cref{lem:general} are satisfied.
    Computing the matrix product reduces to powering $\barA^{k - j}$. Crucially, we can use the fact that matrix powering over floats is in $\L$-uniform $\TC^0$ (\Cref{cor:float-matrix-power}, extending \citealp{mereghetti2000threshold}).
    Finally, $\barA_i, \barB_i, \mathbf C_i,$ and $\mathbf D_i$ can be computed in $\L$-uniform $\TC^0$ because they are constants.
    % We conclude by \Cref{lem:general} that the convolutional form for $M$ can be computed in $\L$-uniform $\TC^0$.
\end{proof}

As S4 satisfies the premises of \Cref{thm:non-gated}, we obtain:

\begin{corollary} \label{cor:s4}
    There exists an $\L$-uniform $\TC^0$ circuit family that computes S4's convolutional form.
\end{corollary}

\subsection{Diagonal SSMs are in $\TC^0$} \label{sec:diagonal}

\begin{theorem}[Diagonal SSM] \label{thm:diagonal}
Let $M$ be a log-precision generalized linear SSM where for $1 \leq i \leq n$:
\begin{compactenum}
    \item the transition matrix $\barA_i$ is diagonal, denoted $\diag(\bar{\mathbf{a}}_i)$ where $\bar{\mathbf a}_i \in \mathbb R^d$;
    \item each of $\bar{\mathbf a}_i, \barB_i, \mathbf C_i$ and $\mathbf D_i$ can be computed in $\L$-uniform $\TC^0$ as a function of $\mathbf x_i$.
\end{compactenum}
Then there exists an $\L$-uniform $\TC^0$ circuit family that computes $M$'s convolutional form.
\end{theorem}

\begin{proof}
    By the first condition, 
    $\prod_i \barA_i = \prod_i \diag(\bar{\mathbf{a}}_i)$.
    Iterated multiplication of diagonal matrices is reducible to several iterated scalar multiplications, placing this product in $\L$-uniform $\TC^0$ (\Cref{cor:iterated-float-product}). 
    The second condition from \Cref{lem:general} is satisfied by assumption. Thus, $M$'s convolutional form is computable in $\L$-uniform $\TC^0$.
\end{proof}

% \subsection{Diagonalizable SSMs are in $\TC^0$} 

% \begin{theorem}[Diagonalizable SSM] \label{thm:diagonal}
%     Let $M$ be a log-precision generalized linear SSM where:
%     \begin{compactenum}
%         \item the set of transition matrices $\{\barA_i\}$ is simultaneously diagonalizable as $\barA_i = \mathbf W \barDiag \mathbf W^{-1}$ for all $1 \leq i \leq n$ where $\bar{\mathbf a}_i \in \mathbb R^d$ and $\mathbf{W}$ is fixed;
%         % For any $i$, $\barA_i = \mathbf W \barDiag \mathbf W^{-1}$, where $\bar{\mathbf a}_i \in \mathbb R^d$.
%         \item for $1 \leq i \leq n$, $\bar{\mathbf a}_i, \barB_i, \mathbf C_i$ and $\mathbf D_i$ can be computed in $\L$-uniform $\TC^0$ as a function of $\mathbf x_i$.
%     \end{compactenum}
%     Then there exists an $\L$-uniform $\TC^0$ circuit family that computes $M$'s convolutional form.
% \end{theorem}

% \begin{proof}
%     When the first condition is satisfied, 
%     $$
%         \prod_i \barA_i = 
%         % \prod_i \mathbf{W}\diag(\bar{\mathbf{a}}_i) \mathbf{W}^{-1} = 
%         \mathbf{W} \left[\prod_i \diag(\bar{\mathbf{a}}_i)\right] \mathbf{W}^{-1}.
%     $$
%     Iterated multiplication of diagonal matrices is reducible to several iterated scalar multiplications, which is in $\L$-uniform $\TC^0$ (\Cref{cor:iterated-float-product}). 
%     Then the product of all $\barA_i$ is the product of three $\L$-uniform $\TC^0$-computable matrices, so is itself $\L$-uniform $\TC^0$-computable.
%     The second condition from \Cref{lem:general} is satisfied by assumption. Thus, the convolutional form for $M$ can be computed in $\L$-uniform $\TC^0$.
% \end{proof}

% \will{Commented out for space}
% \begin{remark}
%     The requirement that the set $\{\barA_i\}$ be \emph{simultaneously diagonalizable} by a shared $\mathbf{W}$, and not just diagonalizable, is necessary to guarantee $\L$-uniform $\TC^0$ computability.
% \end{remark}

% S6 uses a diagonal transition matrix, and we prove in \will{add pointer} that $\barA_i, \barB_i,$ and $\mathbf C_i$ can be computed in $\L$-uniform $\TC^0$ as a function of $x_i$. It thus follows that:

Since S6 satisfies the premises of \Cref{thm:diagonal}, we have:

\begin{corollary} \label{cor:s6-mamba}
    There exists an $\L$-uniform $\TC^0$ circuit family that computes S6's convolutional form (used by Mamba).
\end{corollary}

\begin{proof}
    For the first condition, note that S6's transition matrix $\barA_i$ is defined as $\exp(\delta_i \mathbf{A})$ for a fixed diagonal $\mathbf A$.
    The set of diagonal matrices is closed under scalar multiplication and matrix exponentiation, so $\barA_i$ is also diagonal.
% 
    See \Cref{app:s6} for a proof that the second condition is satisfied by the S6 parameterization.
\end{proof}

\Cref{app:diagonalizable-s6} extends \cref{thm:diagonal} to hold even when $\{\barA_i\}$ are \textbf{simultaneously diagonalizable}, rather than just diagonal. Specifically, we prove the following generalization:

\begin{restatable}[Simultaneously diagonalizable SSM]{theorem}{diagonalizabletheorem} \label{thm:diagonalizable}
Let $\mathbf W$ be a fixed matrix.
Let $M$ be a log-precision generalized linear SSM such that, for $1 \leq i \leq n$,
\begin{compactenum}
    \item the transition matrix $\barA_i$ is computable to log precision by the expression $\mathbf{W}\diag(\bar{\mathbf{a}}_i)\mathbf{W}^{-1}$, where $\bar{\mathbf a}_i \in \mathbb R^d$;
    \item each of $\bar{\mathbf a}_i, \barB_i, \mathbf C_i$ and $\mathbf D_i$ can be computed in $\L$-uniform $\TC^0$ as a function of $\mathbf x_i$.
\end{compactenum}
Then there exists an $\L$-uniform $\TC^0$ circuit family that computes $M$'s convolutional form.
\end{restatable}

This, in turn, allows us to prove that a simultaneously diagonalizable transition matrix generalization of the S6 layer is also in $\L$-uniform $\TC^0$ (\cref{cor:diagonalizable-s6}).


\subsection{Discussion}

\Cref{thm:non-gated,thm:diagonal} establish that common SSM variants, like transformers, can only express solutions to problems in the class $\TC^0$. This means these SSMs cannot solve $\NC^1$-hard problems like evaluating boolean formulas or graph connectivity. In particular, it shows that they are limited as far as their state tracking capabilities as they are unable to compose permutations (solve the $S_5$ word problem):

\begin{corollary}
    Assuming $\TC^0 \neq \NC^1$, no log-precision SSM with the S4 or S6 architecture can solve the word problem for $S_5$ or any other $\NC^1$-hard problem.
\end{corollary}

In contrast, RNNs can easily express $S_5$ via standard constructions that encode finite-state transitions into an RNN \citep{minsky1954neural,merrill-2019-sequential}. This shows that SSMs cannot express some kinds of state tracking and recurrence that RNNs can. This tempers the claim from \citet[Lemma 3.2]{gu2021combining} that SSMs have the expressive power to simulate RNNs, which relied on the assumption that SSMs can have \emph{infinite depth}. In a more realistic setting with a bounded number of layers, our results show SSMs cannot express many state-tracking problems, including those which can be solved by fixed-depth RNNs.

% This appears to contradict the claim from \citet{gu2021combining} that SSMs have the expressive power to simulate RNNs. However, their claim only goes through assuming \emph{infinite depth}. In the realistic setting with a bounded number of layers, our results show that SSMs are not expressive enough to do all kinds of state tracking.

\section{Extending the Expressive Power of SSMs}

We have shown that S4 and S6, despite their seemingly ``stateful'' design, cannot express problems outside $\TC^0$, which includes state tracking like $S_5$. We show how SSMs can be extended to close the gap in expressive power with RNNs, allowing them to express $S_5$.
Two simple extensions can bring about this increase in expressive power, assuming layer input dimension $k > 1$. First, adding a nonlinearity makes the SSM into an RNN, adding expressive power but degrading parallelism. On the other hand, allowing $\barA_i$ to be input-dependent makes the SSM more like a weighted finite automaton (WFA; \citealp{Mohri2009}), adding expressive power while remaining parallelizable.

\subsection{Via Nonlinearities}

One extension to the SSM is to add a nonlinearity, effectively making it an RNN.
We call this an RNN-SSM layer:
\begin{equation*}
    \mathbf h_i = \sgn \left( \barA \mathbf h_{i-1} + \barB x_i \right) .
\end{equation*}
A model with this architecture can solve the $S_5$ word problem when the input dimension $k > 1$:
\begin{theorem} \label{thm:rnn-ssm}
    For any regular language $L \subseteq \Sigma^*$ (including the word problem for $S_5$), there exists a one-layer log-precision RNN-SSM with $k = \abs{\Sigma}$ that recognizes $L$.
\end{theorem}
\begin{proof}
    The standard constructions for simulating automata with RNNs \citep[cf.][]{minsky1954neural,merrill-2019-sequential} apply. The condition $k = \abs{\Sigma}$ comes from needing to represent token types with linearly independent vectors.
\end{proof}

% \begin{corollary}
%     There exists a one-layer log-precision RNN-SSM that solves the word problem for $S_5$ (with a beginning-of-string symbol), and thus log-precision RNN-SSMs cannot be simulated in $\TC^0$.
% \end{corollary}

Adding a nonlinearity to the output of an SSM layer (as in Mamba) is not the same thing as an RNN-SSM. Rather, an RNN-SSM applies the nonlinearity at each recurrent update.
A downside of this approach is that it becomes nonlinear to parallelize the RNN-SSM computation graph with the SCAN algorithm used by linear SSMs \citep{BlellochTR90}. 

\subsection{Via Input-Dependent Transition Matrices} \label{sec:input-dependent}

Another way to get greater expressive power is to let the transition matrix $\barA_i$ be fully input-dependent, as explored by Liquid S4 \citep{hasani2023liquid}.
To illustrate this, we define a minimally different SSM called Input-Dependent S4 (IDS4) that achieves greater expressive power for state tracking.
Let $\pi_{\mathbf A} : \R^k \to \R^{d \times d}$ be some affine transformation where the output vector is interpreted as a $d \times d$ matrix,
and let 
$\barA_i = \pi_{\mathbf A}(\mathbf x_i)$.
% Define
% \begin{equation*}
%     \barA_i = \pi_{\mathbf A}(\mathbf x_i) .
% \end{equation*}
Let $\barB, \mathbf C, \mathbf D$ be fixed (w.r.t.~$i$).
% \will{Technically this is incorrect now. We had $\mathbf x_i$ as a vector} The recurrent form becomes:
% \begin{equation*}
%     \mathbf h_i = [\mathbf I + s_{\barA}(x_i)] \mathbf h_{i-1} + \barB x_i . \label{eq:ids4}
% \end{equation*}
By \Cref{def:generalized-linear-ssm}, the IDS4 convolutional form computes an \emph{iterated product} of non-diagonal, input-dependent matrices:
\begin{equation*}
    \mathbf h_i = \sum_{j=1}^i \left( \prod_{k=j+1}^i \pi_{\mathbf A}(\mathbf x_i) \right) \barB \mathbf x_j .
\end{equation*}

In contrast to matrix powers or iterated products of diagonal matrices, iterated products of \emph{general} matrices cannot be computed in $\TC^0$ \citep{mereghetti2000threshold}.
This means that the arguments from \Cref{thm:non-gated,thm:diagonal} will not go through for IDS4. In fact, we can show IDS4 gains expressive power beyond $\TC^0$:


%%%%%%%%%%%%%%
\begin{figure*}
    \centering
    \includegraphics[width=0.95\textwidth]{figures/group_data.pdf}
    \caption{Minimum number of layers (lower is better) required to attain $>90\%$ validation accuracy on group multiplication problems by sequence length and group. RNN and IDS4 models of constant depth can solve arbitrarily long sequences, while transformer, S4, and Mamba models require depths monotonically increasing in sequence length.}
    \label{fig:group-data}
\end{figure*}
%%%%%%%%%%%%%%



\begin{theorem} \label{thm:wfa-ssm}
    For any regular language $L \subseteq \Sigma^*$ (including the word problem for $S_5$), there exists a one-layer log-precision IDS4 SSM with $k = \abs{\Sigma}$ that recognizes $\$L$, where $\$ \not\in \Sigma$ is a special beginning-of-string symbol.
\end{theorem}

\begin{proof}
    It suffices to show that IDS4 can simulate a deterministic finite automaton (DFA). We do this via a transition monoid construction. For any $w \in \Sigma^*$, let $\delta_w : Q \to Q$ be the function mapping a state to its eventual destination state after $w$ is read from that state. For any DFA, this set of functions forms a finite monoid (the \emph{transition monoid}) under composition,
    following from the Myhill-Nerode theorem \citep{hopcroft2001introduction}.
    Further, each monoid element $\delta_w$ can be represented as a boolean transition matrix, making matrix multiplication isomorphic to monoid composition.
    Computing the transition monoid of a DFA allows recognizing valid words: compute the monoid element for a word by multiplying the elements for its tokens and then check whether the initial state maps to an accepting state.
    % In fact, a standard way to solve monoid word problems (e.g., for $S_5$) with a DFA is simply to construct a DFA whose transition monoid is the monoid of interest.

    Fix a DFA and its transition monoid $\delta$.
    To complete the proof, we show there exists an SSM that, for all $w \in \Sigma^*$, computes $\delta_w$ given input $x=\$w$.
    Let $\barA_i$ be the transition matrix representation of $\delta_{x_i}$. Matrix multiplication is isomorphic to composition of transition monoid elements.
    We view indices in $\mathbf h_i$ as states and define $\barB \$$ as $1$ at the initial state $q_0$ and $0$ elsewhere. For other $\sigma$, let $\barB \sigma = \vec 0$.
    This yields the following convolutional form:
    \begin{equation*}
        \mathbf h_i
        = \left( \prod_{k=2}^{i} \barA_i \right) \mathbf B \$
        \equiv \left( \bigcomp_{k=2}^{i} \delta_{x_k} \right) (q_0) .
    \end{equation*}
    Since $x = \$w$, we conclude that $\mathbf h_{\abs{x}} \equiv \delta_w(q_0)$.
\end{proof}

% \will{==== Liquid S4 ====}
% . The recurrent form of Liquid S4 makes the state transition depend on $\barB x_i$ along with $\barA$:
% $$
% h_i = (\barA + \barB x_i) \mathbf h_{i-1} + \barB x_i.
% $$

% \begin{corollary}
%     There exists a one-layer log-precision WFA-SSM that solves the word problem for $S_5$ (with a beginning-of-string symbol), and thus log-precision WFA-SSMs cannot be simulated in $\TC^0$.
% \end{corollary}

\subsection{Discussion}

\Cref{thm:rnn-ssm,thm:wfa-ssm} show that two minimal extensions of the SSM enable expressive power outside $\TC^0$, allowing the model to solve hard state-tracking problems:

\begin{corollary}
    There exist a one-layer log-precision RNN-SSM and WFA-SSM that express the word problem for $S_5$ (with a beginning-of-string symbol), and these these SSMs cannot be simulated in $\TC^0$.
\end{corollary}

But would these variants of SSMs be feasible to use in practice? Besides expressive power, there are two competing practical concerns that might make these extensions problematic: parallelism and the impact on learning dynamics.

\noindent \textbf{Parallelism.}
To be used effectively in an LLM, a model architecture must be parallelizable on practical hardware. Architectures in $\TC^0$ are parallelizable by design \citep{merrill-sabharwal-2023-parallelism}, but architectures in $\NC^1$ may still be parallelizable to log depth even if they cannot be parallelized to constant depth. For IDS4, the bottleneck would be computing iterated matrix product with a log-depth computation graph. This could be achieved with the SCAN algorithm \citep{BlellochTR90} similar to S4 and S6. In contrast, it is less clear how to parallelize a model with a nonlinearity.


% \noindent\jackson{do we care about this now that LiquidS4 exists?}  \will{I think we should keep this paragraph because it raised a potential concern but argues it \emph{isn't} that important. Better than just not addressing it.}
\textbf{Learning Dynamics.} Another potential concern for IDS4 is that learning dynamics could be degraded. In particular, an iterated product of matrices may lead to vanishing or exploding gradients. However, this is already potentially an issue for the S6 architecture, where the selective gating involves computing an iterated product of scalars.



\section{Can SSMs Learn Permutations in Practice?}

Having established theoretical limitations of SSMs for state tracking, we empirically test
% the types of synthetic tasks that SSMs can learn,
how well SSMs can learn such tasks,
% We first test SSMs and transformers on a task capturing
focusing on the $A_5$ word problem. Since this problem is $\NC^1$-complete and transformers, S4, and Mamba can only express functions in $\TC^0$, these models should require a depth that grows with the input length to solve this problem.

% \subsection{Word Problems}

\noindent \textbf{Task.}
%
We model word problems (see \Cref{sec:word-problems}) as a token-tagging task. Models are given as input a sequence $g_0g_1g_2 \cdots g_n$ drawn from one of $A_5$, $A_4 \times \mathbb{Z}_5$, or $\mathbb{Z}_{60}$. At each step $i$, the label is the product of the first $i$ elements of the sequence. Modeling the problem as a tagging task rather than as a sequence classification task provides the models with more supervision during training, making it as easy as possible to learn the correct function. We tokenize inputs such that each element gets a unique token.

\noindent \textbf{Models.}
We train a transformer as a $\TC^0$ baseline, an RNN that we expect can perform state tracking, and three SSMs: S4 \citep{gu2022efficiently}, Mamba \citep{gu2023mamba}, and IDS4 (\Cref{sec:input-dependent}).
For IDS4, we initialize the affine projection $\alpha$ as a random normal centered around the identity: $\alpha(\mathbf x_i) \sim \mathbf I + \mathcal N(0, \sigma^2)$. 
This ensures that, at initialization, input-dependent transitions tend to propagate the previous state, which we expect to aid learning efficiency.

\noindent \textbf{Experimental Setup.}
We train models on sequences of length $n$ for successively larger values of $n$ and report full-sequence accuracy on a test set.\footnote{We always include all $3600$ pairwise sequences of length $2$ in the training data along with the training split of length-$n$ sequences.} To validate the prediction that SSMs and transformers require growing depth to solve longer $A_5$ word problems, we plot the minimum depth with 90\% test accuracy as a function of input sequence length. 

\noindent \textbf{Results.}
%
\Cref{fig:group-data} shows single-layer RNN and IDS4 models learn the word problem for arbitrarily long sequences for all three groups. In contrast, transformer, S4, and Mamba models require depth monotonically increasing in sequence length to attain good test accuracy for the non-commutative groups. We draw three conclusions from this:

    1. As expected, S4 and Mamba show the same limitations as transformers on the $A_5$ word problem. Longer $A_5$ sequences require deeper models, consistent with these models being in $\TC^0$. In contrast, RNNs (\Cref{thm:rnn-ssm}) and IDS4 (\Cref{thm:wfa-ssm}) can efficiently solve
    % $\NC^1$-complete problems
    % \citep{mckenzie-1991-NC1Viewpoint}
    % including
    the $A_5$ word problem.
    
    2. Transformers, S4, and Mamba require greater depth even for $A_4 \times \mathbb{Z}_5$, which can be theoretically expressed by $\TC^0$ circuits. Although transformer and Mamba models of a given depth perform as good or better on $A_4 \times \mathbb{Z}_5$ as they on $A_5$, they still require increasingly many layers to handle proportionally longer sequences. There are two possible interpretations of this. First, it could be that while these word problems are expressible in $\TC^0$, they cannot be expressed by S4, Mamba, or transformers (which can each likely recognize only a proper subset of $\TC^0$). On the other hand, it is possible that these word problems \emph{are} expressible by transformers, S4, and Mamba but that effectively learning a constant-depth solution is difficult.
    
    3. Despite this limitation, S4 and Mamba appear \emph{empirically better} than transformer at approximate state tracking on the non-commutative tasks. For length-$n$ sequences from $A_4\times\mathbb{Z}_5$ or $A_5$, the transformer requires at least as many (and frequently more) layers as S4 or Mamba.
    
    % For each group and sequence length, the transformer requires at least as many layers as Mamba to solve the task; frequently, the Mamba model can solve the task in fewer layers than is necessary for a transformer model. This disparity is made more impressive by the fact that we models by the hidden state size ($d_\textrm{model}$), which results in SSMs that have roughly half as many parameters as transformers of equal depth. 

% This supports the theoretical analysis showing that SSMs and Transformers exist in the same class of $\TC^0$-simulatable networks. \jackson{Will, is there an analogous citation of yours for RNNs?} \will{\citet{merrill-2019-sequential} characterizes \emph{saturated} RNNs as equivalent to automata} \will{We might want to weaken this last sentence. Maybe say that this finding is consistent with SSMs and transformers being in $\TC^0$... ``support'' is a bit strange because there isn't a strong qualitative difference between TC0/non TC0 conditions}



% upper-bound on computational expressivity for SSMs and Transformers is not the limiting factor in what they can learn to do on state tracking tasks; rather they are more strictly limited by their ability to learn tasks which they can provably solve.

% \will{TODO}
% \jackson{TODO. If we want to shoot for the ICML deadline, it's probably too late to try this, but one way we could emphasize the difference between SSMs/Transformers versus RNNs on S5 data would actually be to look at length generalization as well; the hypothesis being that if the RNNs are actually learning to implement binary multiplication at each time step, then they shouldn't have much difficulty generalizing to OOD lengths since the $h$ vector will only ever need to track the $t_{i-1}$ state, whereas the transformers/SSMs, even if they get to 100\% val-acc, will not generalize well. That might be particularly impactful since the Mamba paper specifically highlights length generalization as a domain where SSMs seem to `solve' the problem (cf. Table 2 on page 11 of the Mamba paper).}
% \will{If we have the trained checkpoints, maybe we can just them on some longer inputs to get a quick and easy experiment? But agree that time might be tight}

\section{Conclusion}

We formally analyzed a family of generalized linear SSMs and showed that, like transformers, common SSM variants including S4 and Mamba can only express computation within the complexity class $\L$-uniform $\TC^0$ of highly parallel computations. This means they cannot solve inherently sequential problems like graph connectivity, boolean formula evaluation, and---of particular interest for state tracking---the permutation composition problem $S_5$. $S_5$ can be naturally expressed by true recurrent models like RNNs and captures the essence of hard state tracking due to its $\NC^1$-completeness. In practice, one-layer RNNs can easily learn a task capturing $S_5$ while linear SSMs require depth growing with the sequence length. These results reveal that S4, Mamba, and related SSMs cannot truly track state: rather, they can only solve simple state-tracking problems for which shallow shortcuts exist \citep{liu2023transformers}.

% , although this comes with potential drawbacks as far as parallelism and learning dynamics.
% and demonstrated a string indexing task where transformers seem to have an expressivity advantage over SSMs.
% In future work, it would be interesting to more thoroughly explore the practical viability of our SSM extensions.
% and investigate the finegrained expressivity differences between transformers and SSMs.
On the other hand, we showed that an input-dependent SSM similar to \citepos{hasani2023liquid} Liquid S4 can both express and learn the $S_5$ word problem, providing evidence that the expressiveness limitations of current SSMs can be overcome.
Ultimately, this line of work could unlock new neural architectures that balance the parallelism of transformers and SSMs with full expressive power for state tracking, enabling LLMs that can benefit from scale while enjoying a greater capacity to reason about games, code, and language.

\section*{Impact Statement}

This paper aims to advance the foundational understanding of state-space architectures for deep learning.
Such work can affect the development and deployment of deep learning models in a variety of ways, which in turn can have societal impacts. However, we find it difficult to meaningfully speculate about or anticipate these downstream impacts here.

\section*{Acknowledgments}
This work benefited from discussions with and valuable feedback from Chris Barker, Stefano Ermon, and Charles Foster.
It was supported in part through the NYU IT High Performance Computing resources, services, and staff expertise. It
was funded by NSF award 1922658, and WM was
supported by an NSF graduate research fellowship, AI2, and Two Sigma.

\bibliographystyle{icml2024}
\begin{thebibliography}{32}
\providecommand{\natexlab}[1]{#1}
\providecommand{\url}[1]{\texttt{#1}}
\expandafter\ifx\csname urlstyle\endcsname\relax
  \providecommand{\doi}[1]{doi: #1}\else
  \providecommand{\doi}{doi: \begingroup \urlstyle{rm}\Url}\fi

\bibitem[Angluin et~al.(2023)Angluin, Chiang, and Yang]{angluin-2023-BRASP}
Angluin, D., Chiang, D., and Yang, A.
\newblock Masked hard-attention transformers and {B}oolean {RASP} recognize exactly the star-free languages, 2023.
\newblock {arXiv}:2310.13897.

\bibitem[Barrington(1989)]{barrington1989bounded}
Barrington, D.~A.
\newblock Bounded-width polynomial-size branching programs recognize exactly those languages in nc1.
\newblock \emph{Journal of Computer and System Sciences}, 38\penalty0 (1):\penalty0 150--164, 1989.
\newblock URL \url{https://www.sciencedirect.com/science/article/pii/0022000089900378}.

\bibitem[Blelloch(1990)]{BlellochTR90}
Blelloch, G.~E.
\newblock Prefix sums and their applications.
\newblock Technical Report CMU-CS-90-190, School of Computer Science, Carnegie Mellon University, November 1990.

\bibitem[Chiang et~al.(2023)Chiang, Cholak, and Pillay]{chiang2023tighter}
Chiang, D., Cholak, P., and Pillay, A.
\newblock Tighter bounds on the expressivity of transformer encoders.
\newblock In \emph{ICML}, 2023.

\bibitem[Feng et~al.(2023)Feng, Zhang, Gu, Ye, He, and Wang]{feng2023towards}
Feng, G., Zhang, B., Gu, Y., Ye, H., He, D., and Wang, L.
\newblock Towards revealing the mystery behind chain of thought: A theoretical perspective.
\newblock In \emph{NeurIPS}, 2023.

\bibitem[Fu et~al.(2023)Fu, Dao, Saab, Thomas, Rudra, and Re]{fu2023hungry}
Fu, D.~Y., Dao, T., Saab, K.~K., Thomas, A.~W., Rudra, A., and Re, C.
\newblock Hungry hungry hippos: Towards language modeling with state space models.
\newblock In \emph{ICLR}, 2023.

\bibitem[Gu \& Dao(2023)Gu and Dao]{gu2023mamba}
Gu, A. and Dao, T.
\newblock Mamba: Linear-time sequence modeling with selective state spaces, 2023.
\newblock arXiv:2312.00752.

\bibitem[Gu et~al.(2021)Gu, Johnson, Goel, Saab, Dao, Rudra, and Re]{gu2021combining}
Gu, A., Johnson, I., Goel, K., Saab, K.~K., Dao, T., Rudra, A., and Re, C.
\newblock Combining recurrent, convolutional, and continuous-time models with linear state space layers.
\newblock In \emph{NeurIPS}, 2021.

\bibitem[Gu et~al.(2022{\natexlab{a}})Gu, Goel, and Re]{gu2022efficiently}
Gu, A., Goel, K., and Re, C.
\newblock Efficiently modeling long sequences with structured state spaces.
\newblock In \emph{ICLR}, 2022{\natexlab{a}}.

\bibitem[Gu et~al.(2022{\natexlab{b}})Gu, Goel, Saab, and Ré]{gu2022blog}
Gu, A., Goel, K., Saab, K., and Ré, C.
\newblock Structured state spaces: Combining continuous-time, recurrent, and convolutional models, January 2022{\natexlab{b}}.
\newblock URL \url{https://hazyresearch.stanford.edu/blog/2022-01-14-s4-3}.
\newblock Blog post accessed January 31, 2024.

\bibitem[Hao et~al.(2022)Hao, Angluin, and Frank]{hao2022ac0}
Hao, S., Angluin, D., and Frank, R.
\newblock Formal language recognition by hard attention transformers: Perspectives from circuit complexity.
\newblock \emph{TACL}, 10:\penalty0 800--810, 2022.

\bibitem[Hasani et~al.(2023)Hasani, Lechner, Wang, Chahine, Amini, and Rus]{hasani2023liquid}
Hasani, R., Lechner, M., Wang, T.-H., Chahine, M., Amini, A., and Rus, D.
\newblock Liquid structural state-space models.
\newblock In \emph{ICLR}, 2023.

\bibitem[Heim(1983)]{heim1983file}
Heim, I.
\newblock File change semantics and the familiarity theory of definiteness.
\newblock \emph{Semantics Critical Concepts in Linguistics}, pp.\  108--135, 1983.

\bibitem[Hesse(2001)]{hesse2001division}
Hesse, W.
\newblock Division is in uniform {$TC^0$}.
\newblock In \emph{International Colloquium on Automata, Languages, and Programming}, pp.\  104--114, 2001.

\bibitem[Hesse et~al.(2002)Hesse, Allender, and Barrington]{Hesse2002UniformCT}
Hesse, W., Allender, E., and Barrington, D. A.~M.
\newblock Uniform constant-depth threshold circuits for division and iterated multiplication.
\newblock \emph{J. Comput. Syst. Sci.}, 65:\penalty0 695--716, 2002.

\bibitem[Hopcroft et~al.(2001)Hopcroft, Motwani, and Ullman]{hopcroft2001introduction}
Hopcroft, J.~E., Motwani, R., and Ullman, J.~D.
\newblock Introduction to automata theory, languages, and computation.
\newblock \emph{ACM SIGACT News}, 32\penalty0 (1):\penalty0 60--65, 2001.

\bibitem[Immerman \& Landau(1989)Immerman and Landau]{immerman1989complexity}
Immerman, N. and Landau, S.
\newblock The complexity of iterated multiplication.
\newblock In \emph{[1989] Proceedings. Structure in Complexity Theory Fourth Annual Conference}, pp.\  104--111, 1989.
\newblock \doi{10.1109/SCT.1989.41816}.

\bibitem[Kim \& Schuster(2023)Kim and Schuster]{kim-schuster-2023-entity}
Kim, N. and Schuster, S.
\newblock Entity tracking in language models.
\newblock In Rogers, A., Boyd-Graber, J., and Okazaki, N. (eds.), \emph{ACL}, July 2023.

\bibitem[Krohn \& Rhodes(1965)Krohn and Rhodes]{krohn1965algebraic}
Krohn, K. and Rhodes, J.
\newblock Algebraic theory of machines. i. prime decomposition theorem for finite semigroups and machines.
\newblock \emph{Transactions of the American Mathematical Society}, 116:\penalty0 450--464, 1965.

\bibitem[Liu et~al.(2023)Liu, Ash, Goel, Krishnamurthy, and Zhang]{liu2023transformers}
Liu, B., Ash, J.~T., Goel, S., Krishnamurthy, A., and Zhang, C.
\newblock Transformers learn shortcuts to automata.
\newblock In \emph{ICLR}, 2023.

\bibitem[Mereghetti \& Palano(2000)Mereghetti and Palano]{mereghetti2000threshold}
Mereghetti, C. and Palano, B.
\newblock Threshold circuits for iterated matrix product and powering.
\newblock \emph{RAIRO-Theor. Inf. Appl.}, 34\penalty0 (1):\penalty0 39--46, 2000.
\newblock \doi{10.1051/ita:2000105}.
\newblock URL \url{https://doi.org/10.1051/ita:2000105}.

\bibitem[Merrill(2019)]{merrill-2019-sequential}
Merrill, W.
\newblock Sequential neural networks as automata.
\newblock In Eisner, J., Gall{\'e}, M., Heinz, J., Quattoni, A., and Rabusseau, G. (eds.), \emph{Proceedings of the Workshop on Deep Learning and Formal Languages: Building Bridges}, Florence, August 2019. ACL.

\bibitem[Merrill \& Sabharwal(2023{\natexlab{a}})Merrill and Sabharwal]{merrill-sabharwal-2023-parallelism}
Merrill, W. and Sabharwal, A.
\newblock The parallelism tradeoff: Limitations of log-precision transformers.
\newblock \emph{TACL}, 11, 2023{\natexlab{a}}.

\bibitem[Merrill \& Sabharwal(2023{\natexlab{b}})Merrill and Sabharwal]{merrill2023logic}
Merrill, W. and Sabharwal, A.
\newblock A logic for expressing log-precision transformers.
\newblock In \emph{NeurIPS}, 2023{\natexlab{b}}.

\bibitem[Merrill \& Sabharwal(2024)Merrill and Sabharwal]{merrill-sabharwal-2024-cot}
Merrill, W. and Sabharwal, A.
\newblock The expressive power of transformers with chain of thought.
\newblock In \emph{ICLR}, 2024.

\bibitem[Minsky(1954)]{minsky1954neural}
Minsky, M.
\newblock Neural nets and the brain-model problem.
\newblock \emph{Unpublished doctoral dissertation, Princeton University, NJ}, 1954.

\bibitem[Mohri(2009)]{Mohri2009}
Mohri, M.
\newblock \emph{Weighted Automata Algorithms}, pp.\  213--254.
\newblock Springer Berlin Heidelberg, Berlin, Heidelberg, 2009.
\newblock ISBN 978-3-642-01492-5.
\newblock \doi{10.1007/978-3-642-01492-5_6}.
\newblock URL \url{https://doi.org/10.1007/978-3-642-01492-5_6}.

\bibitem[Reif \& Tate(1992)Reif and Tate]{reif1992threshold}
Reif, J.~H. and Tate, S.~R.
\newblock On threshold circuits and polynomial computation.
\newblock \emph{SIAM Journal on Computing}, 21\penalty0 (5):\penalty0 896--908, 1992.
\newblock \doi{10.1137/0221053}.
\newblock URL \url{https://doi.org/10.1137/0221053}.

\bibitem[Rush \& Karamcheti(2022)Rush and Karamcheti]{rush2022s4}
Rush, S. and Karamcheti, S.
\newblock The annotated {S4}.
\newblock In \emph{Blog Track at ICLR 2022}, 2022.
\newblock URL \url{https://openreview.net/forum?id=xDaLPsMBZv-}.

\bibitem[Strobl et~al.(2024)Strobl, Merrill, Weiss, Chiang, and Angluin]{strobl2023transformers}
Strobl, L., Merrill, W., Weiss, G., Chiang, D., and Angluin, D.
\newblock What formal languages can transformers express? {A} survey.
\newblock \emph{TACL}, 12, 2024.

\bibitem[Toshniwal et~al.(2021)Toshniwal, Wiseman, Livescu, and Gimpel]{Toshniwal2021ChessAA}
Toshniwal, S., Wiseman, S., Livescu, K., and Gimpel, K.
\newblock Chess as a testbed for language model state tracking.
\newblock In \emph{AAAI}, 2021.

\bibitem[Wang et~al.(2024)Wang, Gangavarapu, Yan, and Rush]{wang2024mambabyte}
Wang, J., Gangavarapu, T., Yan, J.~N., and Rush, A.~M.
\newblock Mambabyte: Token-free selective state space model, 2024.
\newblock arXiv:2401.13660.

\end{thebibliography}


\appendix

% ========== BEGIN APPENDIX CONTENT ==========

%%%%%%%%%%%%%%%
\section{Floating-Point Arithmetic}
\label{app:arithmetic}

Our results use the \textbf{log-precision floating point} model used by \citet{merrill2023logic} to analyze transformers. For some fixed constant $c \in \Z^+$, a $c \log n$ precision float is a tuple $\langle m, e \rangle$ where $m, e$ are signed integers together taking $c \log n$ bits. Using $|x|$ to mean the number of bits used to represent integer $x$, this float represents the value $m \cdot 2^{e - |m| + 1}$.

Unlike for integers, arithmetic operations over log-precision floats are not closed. That is, the product $\phi_1 \times \phi_2$ of two $p$-precision floats is a well-defined number but may not be exactly representable as a $p$-precision float.
It is thus necessary to define approximate versions of these operations when formalizing log-precision floating-point arithmetic.
To this end, \citet{merrill-sabharwal-2023-parallelism} define a natural notion of approximate iterated addition over log-precision floats and show that it is computable in $\L$-uniform $\TC^0$.
We can naturally apply their definition of iterated addition for floats to matrices of floating points, defining iterated summation over matrices of datatype $\D$ as the result of treating the numbers as reals, performing exact arithmetic, and casting the exact output $\phi$ back to $\D$, denoted $\cast_\D(\phi)$. Formally:

\begin{definition}[Iterated $\D$-matrix sum; \citealp{merrill-sabharwal-2023-parallelism}]
    For matrices $\mathbf M_1, \ldots, \mathbf M_n$ over $\D$ with the same size, their \emph{iterated $\D$-sum} is
    \begin{equation*}
        \bigoplus_{i=1}^z \mathbf M_i \, \triangleq \, \cast_\D \left( \sum_{i=1}^z \cast_\R( \mathbf M_i ) \right) .
    \end{equation*}
\end{definition}
Here $\cast_\R$ converts a number in $\D$ to the corresponding real number. $\D$ is implicit in the notations $\cast_\R$ and $\bigoplus$.
Integer addition can be obtained as a special case for $1$-dimensional matrices.
We can also analogously defined iterated summation, which will be necessary for formalizing SSMs:

\begin{definition}[Iterated $\D$-matrix product]
    \label{defn:iterated-D-product}
    For square matrices $\mathbf M_1, \ldots, \mathbf M_z$ over $\D$, their \emph{iterated $\D$-product} is
    \begin{equation*}
        \bigotimes_{i=1}^z \mathbf M_i \, \triangleq \, \cast_\D \left( \prod_{i=1}^z \cast_\R( \mathbf M_i ) \right) .
    \end{equation*}
\end{definition}

% We define matrix powering over $\D$ is define similarly:
% \begin{definition}[$\D$-matrix power]
%     \label{defn:D-matrix-power}
%     For a square matrix $\mathbf M$ over $\D$ and $z \in \Z^+$, \emph{$\D$-matrix power} is defined as
%     \begin{equation}
%         \mathbf M^z \, \triangleq \, \cast_\D ( \cast_\R( \mathbf M )^z ) .
%     \end{equation}
% \end{definition}

\citet{merrill-sabharwal-2023-parallelism} showed that iterated addition from for log-precision floats is in $\L$-uniform $\TC^0$. It naturally follows from their argument that \textbf{iterated addition} over log-precision float matrices is also in $\L$-uniform $\TC^0$.
In general, iterated matrix products are not necessarily computable in $\TC^0$.
However, we extend the arguments of \citet{hesse2001division} and \citet{mereghetti2000threshold} for integers to show that two special cases (iterated scalar multiplication and matrix powering) over log-precision floats are also computable in $\L$-uniform $\TC^0$.

Finally, we define a canonical value for a compositional arithmetic expression over floats that enjoys the associative property.

\begin{definition}[Flattened expression evaluation] \label{def:flattening}
Let $\phi$ be a compositional expression over floats, which may contain alternating sums and products as well as other operations like $\exp$. We define the \emph{canonical value} of $\phi$ as the value returned by the computation graph obtained by flattening all adjacent sums into a single sum (and analogously for products).
\end{definition}

\Cref{def:flattening} has the nice effect of making \Cref{defn:iterated-D-product} associative. The only results that rely on this assumption are our analysis of diagonalizable SSMs in \Cref{app:diagonalizable-s6}.
We also deal with the details of this assumption in \Cref{lem:general}, though the proof there also goes through directly without handling these details.


\subsection{Complexity of Iterated Scalar Multiplication}

The first special case of iterated matrix products we analyze is when the matrices are simply scalars (or, w.l.o.g., diagonal matrices). In this case, the iterated product can be computed in $\L$-uniform $\TC^0$.

\begin{lemma}[Iterated $\D$-product]
    \label{lem:iterated-D-product}
    Let $\phi_1, \ldots, \phi_z \in \D$ be such that $z \leq n$ and each $\phi_i$ can be represented as an $n$-bit integer. If operators $\cast_\D$ and $\cast_\R$ are in $\L$-uniform $\TC^0$, then the iterated $\D$-product $\bigotimes_{i=1}^z \phi_i$ can be computed in $\L$-uniform $\TC^0$.
\end{lemma}

\begin{proof}
    By preconditions of the lemma, we can compute $y_i = \cast_\R( \phi_i )$ for each $i$ in $\L$-uniform $\TC^0$. Since each $\phi_i$ is equivalent to an $n$-bit integer, $y_i$ can be viewed as an $n$-bit integer. The iterated integer product $y = \prod_{i=1}^z y_i$ can be computed with an $\L$-uniform $\TC^0$ circuit \cite{hesse2001division}. Finally, by a precondition of the lemma, we can cast the result back to $\D$, i.e., compute $\cast_\D ( y )$ which equals the iterated $\D$-product $\bigotimes_{i=1}^z \phi_i$, with an $\L$-uniform $\TC^0$ circuit.
\end{proof}

\begin{lemma}[Iterated float product]
    \label{cor:iterated-float-product}
    Let $\phi_1, \ldots, \phi_z$ be $c \log n$ precision floats and $z \leq n$. Then the iterated float product $\bigotimes_{i=1}^z \phi_i$ can be computed in $\L$-uniform $\TC^0$.
\end{lemma}

\begin{proof}
    The idea is to convert (by scaling up) the sequence of $\phi_i$ to another sequence of floats that are all representable as integers, apply \cref{lem:iterated-D-product}, reverse the scaling, and cast the result back to a $c \log n$ precision float.

    Let $e$ be the smallest exponent across all $\phi_i$ and $q = \max \{0, -e\}$. Construct re-scaled floats $\psi_i = \phi_i 2^q$ by adding $q$ to the exponent of $\phi_i$, using up to $c \log n$ additional bits in the exponent if necessary to keep the computation exact. Note that $e$, $q$, and all $\psi_i$ can easily be computed exactly by an $\L$-uniform $\TC^0$ circuit as they involve fixed-arity arithmetic operations. Further, by construction, every $\psi_i$ has a non-negative exponent and thus represents an integer.
    
    % Let $m = c \log n + n^c + 2^q$. Note that $2^q \in [0, n^c]$: hence, $m \leq c \log n + 2n^c$.
    The maximum number representable by each $c \log n$ precision float $\phi_i$ is upper bounded by $2^{n^c}$. Thus, the maximum number representable by each entry $\psi_i$ is $2^{n^c} \times 2^q = 2^{n^c + q}$. Let $m = n^c + q$. It follows that each $\psi_i$ can be equivalently represented as an $m$-bit integer. Further, this integer can be computed by left-shifting the mantissa of $\psi_i$ by a number of bits equal to the value of the exponent of $\psi_i$ (which is non-negative). Finally, this left-shift, and thus the $\cast_\R$ operation over $m$-precision floats, can be easily computed by an $\L$-uniform threshold circuit of size $\poly(m)$. In the other direction, casting from reals to $m$-precision floats can also be easily accomplished by an $\L$-uniform threshold circuit of size $\poly(m)$.

    Observing that $\psi_1, \ldots, \psi_z$ is a sequence of
    % $m$-bit floats,
    floats each representable as an $m$-bit integer,
    we now apply \cref{lem:iterated-D-product} with $\D$ being `float' to conclude that iterated float product $\tau = \bigotimes_{i=1}^z \psi_i$ can be computed by an $\L$-uniform threshold circuit of size $\poly(m)$. Since
    % $m \leq c \log n + 2n^c$,
    $m \leq 2n^c$,
    this circuit is also of size $\poly(n)$.

    Finally, to compute the original iterated float product $\bigotimes_{i=1}^z \phi_i$, we divide $\tau$ by $2^{qz}$. This can be accomplished by subtracting $qz$ from the exponent of $\tau$; again, we do this computation exactly, using up to $(c+1) \log n$ additional bits in the exponent if necessary. We then cast the resulting float back to a $c \log n$ precision float. All this can be done in $\L$-uniform $\TC^0$, finishing the proof that $\bigotimes_{i=1}^z \phi_i$ can be computed in $\L$-uniform $\TC^0$.
\end{proof}


\subsection{Complexity of Matrix Powering}

The second special case we analyze is matrix powering: i.e., a matrix product where all the matrices being powered are the same.
\citet{mereghetti2000threshold} showed that when the datatype $\D$ is $n$-bit integers, one can compute $\mathbf M^n$ in $\TC^0$. We note that their construction also works for computing $\mathbf M^z$ for any $z \leq n, z \in \Z^+$. Further, as they remark, their construction can, in fact, be done in \emph{uniform} $\TC^0$. Specifically, we observe most of their construction involves sums and products of constantly many $n$-bit integers, which can be done in $\L$-uniform $\TC^0$. The only involved step is dividing a polynomial of degree (up to) $n$ by a polynomial of degree (up to) $d-1$ and returning the remainder. It turns out that this ``polynomial division with remainder'' operation can also be performed in $\L$-uniform $\TC^0$ (see Corollary 6.5 of \citealp{Hesse2002UniformCT} and an explanation in \cref{subsec:polynomial-division}). We thus have the following extension of \citeauthor{mereghetti2000threshold}'s result:

\begin{lemma}[Integer matrix power, derived from \citealp{mereghetti2000threshold}]
    \label{lem:integer-matrix-power}
    Let $d \in \Z^+$ be a fixed constant.
    Let $\mathbf M$ be a $d \times d$ matrix over $n$-bit integers and $z \leq n, z \in \Z^+$. Then integer matrix power $\mathbf M^z$ can be computed in $\L$-uniform $\TC^0$.
\end{lemma}

We extend this to matrix powers over $\D$ rather than integers:

\begin{lemma}[$\D$-matrix power]
    \label{lem:D-matrix-power}
    Let $d \in \Z^+$ be a fixed constant.
    Let $\mathbf M$ be a $d \times d$ matrix over a datatype $\D$ with entries equivalently representable as $n$-bit integers. Let $z \leq n, z \in \Z^+$. If operators $\cast_\D$ and $\cast_\R$ are in $\L$-uniform $\TC^0$, then $\D$-matrix power $\mathbf M^z$ can be computed in $\L$-uniform $\TC^0$.
\end{lemma}

\begin{proof}
    By preconditions of the lemma, we can compute $\cast_\R( \mathbf M )$ in $\L$-uniform $\TC^0$. Since the entries of $\mathbf M$ are equivalent to $n$-bit integers, $\cast_R( \mathbf M )$ can be viewed as a $d \times d$ integer matrix of $n$-bit integers. By \cref{lem:integer-matrix-power}, we can compute $\cast_R( \mathbf M )^z$ using an $\L$-uniform $\TC^0$ circuit. Finally, by a precondition of the lemma, we can cast the result back to $\D$, i.e., compute $\cast_\D(\cast_\R( \mathbf M )^z)$ which equals $\mathbf M^z$, with an $\L$-uniform $\TC^0$ circuit.
\end{proof}

\begin{lemma}[Float matrix power]
    \label{cor:float-matrix-power}
    Let $d, c \in \Z^+$ be fixed constants.
    Let $\mathbf M$ be a $d \times d$ matrix over $c \log n$ precision floats. Let $z \leq n, z \in \Z^+$. Then float matrix power $\mathbf M^z$ can be computed in $\L$-uniform $\TC^0$.
\end{lemma}

\begin{proof}
    The idea is to convert (by scaling up) $\mathbf M$ to another float matrix all whose entries are representable as integers, apply \cref{lem:D-matrix-power}, reverse the scaling, and cast the result back to $c \log n$ precision floats.

    Let $e$ be the smallest exponent across all float entries of $\mathbf M$ and $q = \max \{0, -e\}$. Construct a re-scaled float matrix $\tilde{\mathbf M} = \mathbf M 2^q$ by adding $q$ to the exponent of every entry of $\mathbf M$, using up to $c \log n$ additional bits in the exponent if necessary to keep the computation exact. Note that $e$, $q$, and $\tilde{\mathbf M}$ can easily be computed exactly by an $\L$-uniform $\TC^0$ circuit as they involve fixed-arity arithmetic operations. Further, by construction, $\tilde{\mathbf M}$ has non-negative exponents in all its float entries. Thus, every entry of $\tilde{\mathbf M}$ represents an integer.
    
    % Let $m = c \log n + n^c + 2^q$. Note that $2^q \in [0, n^c]$: hence, $m \leq c \log n + 2n^c$.
    The maximum number representable by each $c \log n$ precision float in $\mathbf M$ is upper bounded by $2^{n^c}$. Thus, the maximum number representable by each entry of $\tilde{\mathbf M}$ is $2^{n^c} \times 2^q = 2^{n^c + q}$. Let $m = n^c + q$. It follows that each entry $\phi$ of $\tilde{\mathbf M}$ can be equivalently represented as an $m$-bit integer. Further, this integer can be computed by left-shifting the mantissa of $\phi$ by a number of bits equal to the value of the exponent of $\phi$ (which is non-negative). Finally, this left-shift, and thus the $\cast_\R$ operation over $m$-precision floats, can be easily computed by an $\L$-uniform threshold circuit of size $\poly(m)$. In the other direction, casting from reals to $m$-precision floats can also be easily accomplished by an $\L$-uniform threshold circuit of size $\poly(m)$.

    Note that $2^q \in [0, n^c]$ and hence $m \in [n^c, 2n^c]$. In particular, $m \geq n$. Thus $z \leq n$ (a precision) implies $z \leq m$. Observing that $\tilde{\mathbf M}$ is a matrix of
    % $m$-bit floats,
    floats each representable as an $m$-bit integer,
    we now apply \cref{lem:D-matrix-power} with $\D$ being `float' to conclude that float matrix power $\tilde{\mathbf M}^z$ can be computed by an $\L$-uniform threshold circuit of size $\poly(m)$. Since
    % $m \leq c \log n + 2n^c$,
    $m \leq 2n^c$,
    this circuit is also of size $\poly(n)$.

    Finally, to compute $\mathbf M^z$, we first divide each entry of $\tilde{\mathbf M}^z$ by $2^{qz}$. This can be accomplished by subtracting $qz$ from the exponent of each entry of $\tilde{\mathbf M}$; again, we do this computation exactly, using up to $(c+1) \log n$ additional bits in the exponent if necessary. We then cast all entries of the resulting matrix back to $c \log n$ precision floats. All this can be done in $\L$-uniform $\TC^0$, finishing the proof that $\mathbf M^z$ can be computed in $\L$-uniform $\TC^0$.
\end{proof}


\subsection{$\L$-Uniformity of Polynomial Division in $\TC^0$}
\label{subsec:polynomial-division}

\citet{Hesse2002UniformCT} state that polynomial division is in $\L$-uniform $\TC^0$ in Corollary 6.5. For historical reasons, this claim is preceded by weaker claims in older papers. We briefly clarify this situation to help understand why the stronger claim is valid.

\citet{reif1992threshold} establish that polynomial division can be performed in $\P$-uniform $\TC^0$, whereas we state our results for $\L$-uniform $\TC^0$, which is a smaller class. However, the only issue preventing the polynomial division result from originally going through in the $\L$-uniform case is that, at the time of \citeauthor{reif1992threshold}'s publication, it was not known whether integer division and iterated integer multiplication are computable in $\L$-uniform $\TC^0$. However, \citet{hesse2001division} later proved exactly this. Combining the two results, Theorem 3.2 of \citet{reif1992threshold} goes through even with $\L$-uniformity (not just $\P$-uniformity). Its Corollary 3.3 then allows us to conclude that integer polynomial division can be solved by $\L$-uniform $\TC^0$ circuits because the output of integer polynomial division is an analytic function whose Taylor expansion has a finite number of terms \citep{reif1992threshold}.

% Alternatively and more directly, it may be possible to strengthen the $\L$-uniform $\NC^1$ reduction from polynomial division to iterated multiplication given by \citet{eberly1989veryFast} to an $\L$-uniform $\TC^0$ reduction given the new knowledge that iterated multiplication is in $\L$-uniform $\TC^0$ \citep{hesse2001division}, since the reduction itself uses iterated multiplication circuits as a black box \citep{eberly1989veryFast}.

%%%%%%%%%%%%%%%

\section{S6 Parameterization} \label{app:s6}

To justify that the S6 architecture used by Mamba is computable in $\TC^0$, we justify that $\barA_i, \barB_i, \mathbf C_i, \mathbf D_i$ can be computed as a function of $\mathbf x_i$ in $\TC^0$.

We begin by summarizing how exactly is S6 parameterized. S6 first defines continuous-time parameters:
\begin{compactenum}
    \item $\mathbf A$ is a fixed, diagonal matrix that is invertible (each $a_{ii} \neq 0$);
    \item $\mathbf B_i = \pi_{\mathbf B}(\mathbf x_i)$ is computed via a projection;
    \item $\mathbf C_i = \pi_{\mathbf C}(\mathbf x_i)$ is computed via a projection;
    \item $\mathbf D_i = \mathbf I$ .
\end{compactenum}
Next, we need to discretize the matrices $\mathbf A$ and $\mathbf B$.
S6 does this using an input-dependent discretization factor $\delta_i$:
\begin{equation*}
    \delta_i = \mathrm{softplus}(\delta + \pi_\delta(\mathbf x_i)) .
\end{equation*}
The discretized matrices are then defined as:
\begin{align*}
    \barA_i &= \exp(\delta_i \mathbf A) \\
    \barB_i &= (\delta_i \mathbf A)^{-1} \left( \barA_i - \mathbf I \right) \delta_i \mathbf B_i .
\end{align*}

It is clear to see that the diagonalizability condition of \Cref{thm:diagonal} is satisfied because $\barA_i$ itself is diagonal. Additionally, all the relevant matrices can be computed in $\TC^0$.

\begin{proposition}
    $\barA_i, \barB_i, \mathbf C_i,$ and $\mathbf D_i$ can all be computed as functions of $\mathbf x_i$ in $\L$-uniform $\TC^0$.
\end{proposition}

To prove this, observe that
$\mathbf A, \mathbf B_i, \mathbf C_i, \mathbf D_i$ can all be computed in $\L$-uniform $\TC^0$ because they are either constants or linear transformations of $\mathbf x_i$. To justify that $\barA_i$ and $\barB_i$ can be computed in $\L$-uniform $\TC^0$, we just need to justify that we can invert diagonal matrices and compute $\mathrm{softplus}$ and $\exp$ in $\L$-uniform $\TC^0$.

\begin{lemma} \label{lem:invert_diagonal}
    Diagonal matrices over log-precision floats can be inverted in $\L$-uniform $\TC^0$.
\end{lemma}

\begin{proof}
    Inverting a diagonal matrix just involves forming the reciprocal along the diagonal. Scalar reciprocals can be approximated to error at most $2^{-n^c}$ (for any $c$) in $\TC^0$ \citep{Hesse2002UniformCT}. This means we can compute the reciprocal of a log-precision float (cf. \Cref{app:arithmetic}) exactly up to log precision.
\end{proof}

In \Cref{sec:nonlinearities}, we show that we can compute the nonlinearities $\exp$ and $\softplus$ over a bounded domain in $\TC^0$.

\section{Diagonalizable SSMs} \label{app:diagonalizable-s6}

We extend \cref{thm:diagonal} to cover the case when the SSMs transition matrices are \emph{simultaneously diagonalizable}, rather than just diagonal. This requires us to note that when working with log-precision floating point representations of matrices, a diagonal matrix $\mathbf{A}$ and its diagonalized decomposition $\mathbf{W}\diag(\mathbf{a})\mathbf{W}^{-1}$ are numerically substitutable.

\diagonalizabletheorem*

\begin{proof}
    When the first condition is satisfied, the following equality holds over log-precision floats:
    \begin{equation*}
        \prod_i \barA_i = \prod_i \left( \mathbf{W}\diag(\bar{\mathbf{a}}_i)\mathbf{W}^{-1} \right) .
    \end{equation*}
    By the associativity of $\mathbb D$-matrix products, we can remove the parentheses to get
    \begin{align*}
        \prod_i \barA_i
        &= \prod_i\mathbf{W}\diag(\bar{\mathbf{a}}_i)\mathbf{W}^{-1} \\
        &= \mathbf{W} \left[\prod_i \diag(\bar{\mathbf{a}}_i)\right]\mathbf{W}^{-1}.
    \end{align*}
    Iterated multiplication of diagonal matrices is reducible to several iterated scalar multiplications, which is in $\L$-uniform $\TC^0$ (\Cref{cor:iterated-float-product}). 
    Then the product of all $\barA_i$ is the product of three $\L$-uniform $\TC^0$-computable matrices, so is itself $\L$-uniform $\TC^0$-computable.
    The second condition from \Cref{lem:general} is satisfied by assumption. Thus, the convolutional form for $M$ can be computed in $\L$-uniform $\TC^0$.
\end{proof}

\subsection{Diagonalizable S6}

We can define an extension of S6 which satisfies these conditions to show that it is also in $\L$-uniform $\TC^0$.

\begin{definition}
    Diagonalizable S6 has continuous-time parameters:
    \begin{compactenum}
        \item $\mathbf{A}$ is a fixed matrix diagonalizable as $\mathbf{W}\diag(\mathbf{a})\mathbf{W}^{-1}$ that is invertible (each $a_{ii} \neq 0$);
        \item $\mathbf{B}_i = \pi_\mathbf{B}(\mathbf{x}_i)$ is computed via a projection;
        \item $\mathbf{C}_i = \pi_\mathbf{C}(\mathbf{x}_i)$ is computed via a projection;
        \item $\mathbf{D} = \mathbf{I}$.
    \end{compactenum}
    As in the standard S6, the discretization of $\mathbf{A}$ and $\mathbf{B}$ is done by an input-dependent discretization factor $\delta_i$:
    $$
        \delta_i = \mathrm{softplus}(\delta + \pi_\delta(\mathbf{x}_i)).
    $$
    The discretized matrices are then defined as
    \begin{align*}
        \barA_i &= \exp(\delta_i \mathbf{A}), \\
        \barB_i &= (\delta_i\mathbf{A})^{-1}(\barA_i - \mathbf{I})\delta_i\mathbf{B}_i.
    \end{align*}
\end{definition}

To prove that $\barA_i$ and $\barB_i$ have the necessary properties, we first introduce some lemmas dealing with matrix-valued functions of diagonalizable matrices.

\begin{lemma}
    If a matrix $\mathbf{A}$ is diagonalizable, then we can substitute its diagonalized decomposition $\mathbf{W}\diag(\mathbf{a})\mathbf{W}^{-1}$ in a computation graph over log-precision floats involving $\mathbf{A}$ without incurring any meaningful error.
\end{lemma}
\begin{proof}
    Let $\mathbf{A}$ be diagonalizable. Then there exists invertible $\mathbf{W}$ and diagonal $\diag(\mathbf{a})$ such that $\mathbf{A} = \mathbf{W}\diag(\mathbf{a})\mathbf{W}^{-1}$. Note that the product of a fixed number of matrices is in $\L$-uniform $\TC^0$, and so the first $c \log n$ bits of $\mathbf{A}$ and $\mathbf{W}\diag(\mathbf{a})\mathbf{W}^{-1}$ are identical.
\end{proof}

\begin{lemma} \label{lem:factor_scalar}
    Let $\mathbf{A}$ be diagonalizable as $\mathbf{W}\diag(\mathbf{a})\mathbf{W}^{-1}$, where $\mathbf{a} \in \mathbb{R}^d$.
    Then $c\cdot\mathbf{A}$ is simultaneously diagonalizable with $\mathbf{A}$ via $c \cdot \mathbf{A} = \mathbf{W}c\cdot\diag(\mathbf{a})\mathbf{W}^{-1}$.
\end{lemma}
\begin{proof}
    Scalar multiplication commutes around matrix multiplication.
\end{proof}

\begin{lemma} \label{lem:factor_exp}
    Let $\mathbf{A}$ be diagonalizable as $\mathbf{W}\diag(\mathbf{a})\mathbf{W}^{-1}$, where $\mathbf{a} \in \mathbb{R}^d$. Then $\exp(\mathbf{A}) = \mathbf{W} \exp(\diag(\mathbf{a})) \mathbf{W}^{-1}$.
\end{lemma}

\begin{proof}
    The matrix exponential is defined as a power series, so for diagonalizable $\mathbf{A}$ it follows that
    \begin{align*}
        \exp(\mathbf{A}) &= \exp(\mathbf{W}\diag(\mathbf{a})\mathbf{W}^{-1}) \\
        &= \sum_{k=0}^\infty \frac{1}{k!} (\mathbf{W}\diag(\mathbf{a})\mathbf{W}^{-1})^k \\
        &= \sum_{k=0}^\infty \frac{1}{k!} \mathbf{W}\diag(\mathbf{a})^k\mathbf{W}^{-1} \\
        &= \mathbf{W}\left(\sum_{k=0}^\infty \frac{1}{k!}\diag(\mathbf{a})^k\right)\mathbf{W}^{-1} \\
        &= \mathbf{W}\exp(\diag(\mathbf{a}))\mathbf{W}^{-1}. \tag*{\qedhere}
    \end{align*}
\end{proof}

The expressions in \Cref{lem:factor_exp} are equivalent not just over real numbers but also over log-precision floats. This is because we know both expressions can be approximated in $\TC^0$ with error at most $2^{-n^c}$, which means the $c \log n$ bits of the approximation must be equivalent.
% \will{This should be sufficient. Mathematically, the expressions are equivalent. We also know we can compute $\exp(\mathbf{A})$ in $\TC^0$ exactly up to $c \log n$ bits and we can compute $\mathbf{W}\exp(\diag(\mathbf{a}))\mathbf{W}^{-1}$ exactly up to $c \log n$ bits as well. Therefore, these $c \log n$-bit approximations must be the same.} \jackson{maybe lemmatize out?}

\begin{lemma} \label{lem:invert_diagonalizable}
Diagonalizable matrices over log-precision floats can be inverted in $\L$-uniform $\TC^0$.
\end{lemma}

\begin{proof}
    Let $\mathbf{A} = \mathbf{W}\diag(\mathbf{a})\mathbf{W^{-1}}$. Then
    $\mathbf{A}^{-1} = \mathbf{W}^{-1}\diag(\mathbf{a})^{-1}\mathbf{W}$. We are guaranteed that each of these matrices exists, and furthermore by \Cref{lem:invert_diagonal} we know that $\diag(\mathbf{a})^{-1}$ is computable in $\L$-uniform $\TC^0$. Their product, involving a finite number of additions and multiplies, is also computable in $\L$-uniform $\TC^0$.
\end{proof}

\begin{proposition}
    $\barA_i$ and $\barB_i$ can be computed as functions of $\mathbf{x}_i$ in $\L$-uniform $\TC^0$.
\end{proposition}

\begin{proof}
    We first show that $\barA_i$ is $\L$-uniform $\TC^0$ computable. By definition,
    $$
        \barA_i = \exp(\delta_i \mathbf{A}).
    $$
    By \Cref{cor:exp-log-softplus}, $\delta_i$ is computable in $\L$-uniform $\TC^0$.
    The product $\delta_i\mathbf{A}$ is simultaneously diagonalizable with $\mathbf{A}$ so
    \begin{align*}
        \barA_i &=\exp(\mathbf{W}\delta_i \diag(\mathbf{a})\mathbf{W}^{-1}) \tag{\Cref{lem:factor_scalar}} \\
        &= \mathbf{W}\exp(\diag(\mathbf{a}))\mathbf{W}^{-1}. \tag{\Cref{lem:factor_exp}}
    \end{align*}
    Since the exponential of scalars is $\L$-uniform $\TC^0$ computable by \Cref{cor:exp-log-softplus}, then $\barA_i$ is as well.

    Turning to $\barB_i$, note that the term $(\delta_i\mathbf{A})^{-1}$ is $\L$-uniform $\TC^0$ computable by \Cref{lem:invert_diagonalizable} since $\delta_i\mathbf{A}$ is diagonalizable. Since $\barA_i$ is $\L$-uniform $\TC^0$ computable, the difference $\barA_i - \mathbf{I}$ is as well. Then every term in 
    $$
    \barB_i = (\delta_i\mathbf{A})^{-1}(\barA_i - \mathbf{I})\delta_i\mathbf{B}_i
    $$
    is $\L$-uniform $\TC^0$ computable, and so their product is as well.
\end{proof}

\begin{remark}
    Since $\mathbf{C}_i$ and $\mathbf{D}_i$ are unchanged between the standard and diagonalizable versions of S6, the proofs of their computability as functions of $\mathbf{x}_i$ in $\L$-uniform $\TC^0$ pass through from \Cref{app:s6}.
\end{remark}

\begin{corollary}\label{cor:diagonalizable-s6}
    There exists an $\L$-uniform $\TC^0$ circuit family that computes Diagonalizable S6's convolutional form.
\end{corollary}

\begin{proof}
    Note that since $\mathbf{A} = \mathbf{W}\diag(\mathbf{a})\mathbf{W}^{-1}$ is fixed the set of transition matrices $\{\barA_i\}$ is simultaneously diagonalizable via $\mathbf{W}$ for all $i$.
    
    Then Diagonalizable S6 meets the conditions for \Cref{thm:diagonalizable}.
\end{proof}

\section{Nonlinearities in $\L$-Uniform $\TC^0$} \label{sec:nonlinearities}.

The parameterization of SSMs (and transformers) involves computing nonlinearities like $\exp$ and $\softplus$. We leverage existing circuit complexity results \citep{reif1992threshold} to show that, in general, any well-behaved nonlinearity should be computable in $\L$-uniform $\TC^0$ when used in conjunction with pre- or post-layer norm.



\begin{lemma}[Adapts Corollary 3.3, \citealp{reif1992threshold}] \label{lem:nonlinearity}
    Let $X = (-B, B)$ be a bounded interval.
    Let $f$ be a function over $X$ with a convergent Taylor series:
    \begin{equation*}
        f(x) = \sum_{n=0}^\infty \frac{a_n}{b_n} (x - x_0)^n ,
    \end{equation*}
    where $a_n, b_n$ are integers with magnitude at most $2^{n^{O(1)}}$ computable in $L$-uniform $\TC^0$.
    Then $f$ can be approximated over $X$ by $\L$-uniform $\TC^0$ circuits to log precision (error at most $2^{-n^c}$ for any $c \geq 1$).
\end{lemma}

\begin{proof}
    \citet{reif1992threshold} give a proof when $X = (-1, 1)$. We generalize to $X = (-B, B)$, assuming w.l.o.g. $B = 2^k$. The idea is to transform $f$ to have domain $(-1, 1)$ via
    \begin{equation*}
        g(x) = f(Bx) .
    \end{equation*}
    Then, we can apply Corollary 3.3 of \citet{reif1992threshold} to approximate $g$ with error at most $2^{-n^c}$. \citet{reif1992threshold} state their result for $\P$-uniform $\TC^0$, but through advances in circuit complexity since the time of publication~(\Cref{subsec:polynomial-division}), their construction naturally applies for $\L$-uniform $\TC^0$ as well.
    
    To approximate $f$, compute $z = x / B$, which can be done exactly since $B = 2^k$. We conclude by computing $g(z) = f(x)$, which, as established, has error at most $2^{-n^c}$.
\end{proof}

% Since $\softplus(x) = \log \left( 1 + \exp(x) \right)$, $\softplus$ can also be approximated to log precision.

Because of pre- and post-norm layers, the elements of $\mathbf x_i$ in an SSM will remain in a bounded domain $(-B, B)$. Thus, the following lemma shows we can compute them:

\begin{corollary} \label{cor:exp-log-softplus}
    The pointwise nonlinearities $\exp$, $\log$, and $\softplus$ are computable over $(-B, B)$ in $\L$-uniform $\TC^0$.
\end{corollary}

\begin{proof}
    % For $\exp$ and $\log$, note that the use of pre- or post-layer norm means that the boundedness condition of \cref{lem:nonlinearity} is satisfied.
    By \citet[Corollary 3.3]{reif1992threshold} know that the Taylor series for $\exp$ and $\log$ is convergent with $a_n,b_n$ computable in $\L$-uniform $\TC^0$. Then $\exp$ and $\log$ are themselves computable in $\L$-uniform $\TC^0$.

    Since $\softplus(x) = \log \left( 1 + \exp(x) \right)$ is a fixed composition of $L$-uniform $\TC^0$-computable functions, it too is computable in $L$-uniform $\TC^0$.
\end{proof}
% ========== END APPENDIX CONTENT ==========


\end{document}
